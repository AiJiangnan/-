\section{误差理论与数据处理\scite{1}}
\subsection{误差的基本性质与处理}
\subsubsection{随机误差}
当我们对同一个测量值进行多次的等精度重复测量时,可以得到一系列的不同的测量值,这些测量值多多少少都会存在误差,它们的出现是没有确定的规律的,但对于它们总体而言,却有统计的规律性。

当测量列中不包括系统误差和粗大误差时,随机误差的分布可以是正态分布,也可以是其他分布,比如,均匀分布、三角形分布、$ \chi^2 $分布等,而大多数随机误差都服从正态分布。

假设被测量的真值为$ L_0 $,而测得值为$ l_i(i=1,2,...,n) $,则测量数据的随机误差$ \delta_i $为\[ \delta_i=l_i-L_0 \]

正态分布的分布密度$ f(\delta) $与分布函数$ F(\delta) $为\[ f(\delta)=\frac{1}{\sigma\sqrt{2\pi}}e^{-\delta^2/(2\sigma^2)} \]\[ F(\delta)=\frac{1}{\sigma\sqrt{2\pi}}\int_{-\infty}^{\delta}e^{-\delta^2/(2\sigma^2)}d\delta \]
式中:$ \sigma $为总体标准差,$ e $为自然对数底。
\begin{enumerate}
	\item \textbf{算术平均值}
	
	在一系列的测量值中,被测量的$ n $个测得值的代数和除以$ n $而得的值称为算术平均值。设$ l_1,l_2,...,l_n $为$ n $次测量所得的值,则算术平均值$ \bar{x} $为\[ \bar{x}=\frac{l_1+l_2+...+l_n}{n}=\frac{1}{n}\sum_{i=1}^{n}l_i \]
	当测量次数无限大时,算术平均值被认为是最接近于真值的。
	\item \textbf{标准差}
	
	\textbf{单次测量的标准差 } 同一个被测量,在相同条件下,测量列$ l_1,l_2,...,l_n $中单次测量的标准差是表征同一被测量$ n $次测量结果分散性的参数,并按下式计算。
	\[ \sigma=\sqrt{\frac{\sum\limits_{i=1}^{n}(l_i-L_0)^2}{n}}=\sqrt{\frac{\sum\limits_{i=1}^{n}\delta_i^2}{n}} \]
	式中:$ n $为测量次数(应充分大);$ \delta_i $为第$ i $个测量值所对应的随机误差,即测量值与被测量值的真值之差。
	
	对于标准差恒等的测量,我们把它定义为等精度测量,对于相同的测量条件下所做的重复测量均为等精度测量。反之,则属于不等精度测量。
	
	对同一被测量,在相同测量条件下,进行有限次测量得测量列$ l_1,l_2,...,l_n $,则单次测量标准差的估计值为\[ s=\sqrt{\frac{\sum\limits_{i=1}^{n}v_i^2}{n-1}} \]
	这也被叫做贝塞尔公式。
	
	\textbf{算术平均值的标准差 } 算术平均值的标准差$ \bar{s} $则是表征同一被测量的各个独立测量列算术平均值分散的参数。\[ \bar{s}=\frac{s}{\sqrt{n}} \]
	\item \textbf{测量的极限误差}
	
	\textbf{单次测量的极限误差 } 根据概率论知识,已知正态分布曲线可得:\[ p=\int_{-\infty}^{+\infty}f(\delta)d\delta=\int_{-\infty}^{+\infty}\frac{1}{\sigma\sqrt{2\pi}}e^{-\frac{\delta^2}{2\sigma^2}}d\delta=1 \]
	由此,误差落在区间$ [-\delta,+\delta] $之间的概率为:\[ p=\int_{-\delta}^{+\delta}f(\delta)d\delta=\int_{-\delta}^{+\delta}\frac{1}{\sigma\sqrt{2\pi}}e^{-\frac{\delta^2}{2\sigma^2}}d\delta \]
	将上式进行变量转换,设\[ t=\frac{\delta}{\sigma},dt=\frac{d\delta}{\sigma} \]
	经变换,上式成为\[ p=\frac{1}{\sqrt{2\pi}}\int_{-t}^{+t}e^{-\frac{t^2}{2}}dt=\frac{2}{\sqrt{2\pi}}\int_{0}^{+t}e^{-\frac{t^2}{2}}dt=2\Phi(t) \]
	\[ \Phi(t)=\frac{1}{\sqrt{2\pi}}\int_{0}^{+t}e^{-\frac{t^2}{2}}dt \]
	若某随机误差在$ \pm t\sigma $范围内出现的概率为$ 2\Phi(t) $,则超出的概率为\[ \alpha=1-2\Phi(t) \]
	不同的$ t $时超出$ |\delta| $的概率是不同的,取不同的$ t $值时,极限误差可用下式表示:\[ \delta_{lim}x=\pm t\delta \]
	这可以得出算术平均值的极限误差:\[ \delta_{lim}\bar{x}=\pm t_a\sigma_{\bar{x}} \]
	式中:$ t_a $为置信系数;$ \alpha $为超出极限误差的概率;$ \sigma_{\bar{x}} $为$ n $次测量的算术平均值标准差。
	\item \textbf{权}
	
	在等精度测量中,各个测得的值我们认为是同样可靠的,用所有值的算术平均值作为最后的测量结果。但是在不等精度测量中,各个测量值的可靠程度是不一样的,所以我们用权来说明测量的可靠程度。我们根据算术平均值标准差以及测量的次数来确定权,假设同一测量量有$ m $组不等精度的测量结果,可表示为\[ p_1:p_2:...:p_m=\frac{1}{\sigma_{\bar{x_1}}^2}:\frac{1}{\sigma_{\bar{x_2}}^2}:...:\frac{1}{\sigma_{\bar{x_m}}^2} \]
	\item \textbf{加权算术平均值}
	
	
\end{enumerate}
\subsubsection{系统误差}
\subsubsection{粗大误差}
\subsubsection{等精度测量数据的误差分析}
\subsubsection{不等精度测量数据的误差分析}
\subsection{误差的合成}
\subsection{测量不确定度}
\subsection{最小二乘法处理}
\subsubsection{等精度测量线性参数最小二乘法处理}
\subsubsection{不等精度测量线性参数最小二乘法处理}
\subsection{回归分析}
\subsubsection{一元线性回归}
\subsubsection{一元非线性回归}
\subsubsection{多元线性回归}