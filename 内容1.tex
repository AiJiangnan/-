\section{误差理论与数据处理\scite{1}}
\subsection{误差的基本性质与处理}
\subsubsection{随机误差}
当我们对同一个测量值进行多次的等精度重复测量时,可以得到一系列的不同的测量值,这些测量值多多少少都会存在误差,它们的出现是没有确定的规律的,但对于它们总体而言,却有统计的规律性。

当测量列中不包括系统误差和粗大误差时,随机误差的分布可以是正态分布,也可以是其他分布,比如,均匀分布、三角形分布、$ \chi^2 $分布等,而大多数随机误差都服从正态分布。

假设被测量的真值为$ L_0 $,而测得值为$ l_i(i=1,2,...,n) $,则测量数据的随机误差$ \delta_i $为\[ \delta_i=l_i-L_0 \]

正态分布的分布密度$ f(\delta) $与分布函数$ F(\delta) $为\[ f(\delta)=\frac{1}{\sigma\sqrt{2\pi}}e^{-\delta^2/(2\sigma^2)} \]\[ F(\delta)=\frac{1}{\sigma\sqrt{2\pi}}\int_{-\infty}^{\delta}e^{-\delta^2/(2\sigma^2)}d\delta \]
式中:$ \sigma $为总体标准差,$ e $为自然对数底。
\begin{enumerate}
	\item \textbf{算术平均值}
	
	\qquad 在一系列的测量值中,被测量的$ n $个测得值的代数和除以$ n $而得的值称为算术平均值\footnote{《误差理论与数据处理》第22页算术平均值的定义,详情见参考文献[1]}。设$ l_1,l_2,...,l_n $为$ n $次测量所得的值,则算术平均值$ \bar{x} $为\[ \bar{x}=\frac{l_1+l_2+...+l_n}{n}=\frac{1}{n}\sum_{i=1}^{n}l_i \]
	当测量次数无限大时,算术平均值被认为是最接近于真值的。
	\item \textbf{标准差}
	
	\textbf{单次测量的标准差 } 同一个被测量,在相同条件下,测量列$ l_1,l_2,...,l_n $中单次测量的标准差是表征同一被测量$ n $次测量结果分散性的参数,并按下式计算。
	\[ \sigma=\sqrt{\frac{\sum\limits_{i=1}^{n}(l_i-L_0)^2}{n}}=\sqrt{\frac{\sum\limits_{i=1}^{n}\delta_i^2}{n}} \]
	式中:$ n $为测量次数(应充分大);$ \delta_i $为第$ i $个测量值所对应的随机误差,即测量值与被测量值的真值之差。
	
	\qquad 对于标准差恒等的测量,我们把它定义为等精度测量,对于相同的测量条件下所做的重复测量均为等精度测量。反之,则属于不等精度测量。
	
	\qquad 对同一被测量,在相同测量条件下,进行有限次测量得测量列$ l_1,l_2,...,l_n $,则单次测量标准差的估计值为\[ s=\sqrt{\frac{\sum\limits_{i=1}^{n}v_i^2}{n-1}} \]
	这也被叫做贝塞尔公式。
	
	\textbf{算术平均值的标准差 } 算术平均值的标准差$ \bar{s} $则是表征同一被测量的各个独立测量列算术平均值分散的参数。\[ \bar{s}=\frac{s}{\sqrt{n}} \]
	\item \textbf{测量的极限误差}
	
	\textbf{单次测量的极限误差 } 根据概率论知识,已知正态分布曲线可得:\[ p=\int_{-\infty}^{+\infty}f(\delta)d\delta=\int_{-\infty}^{+\infty}\frac{1}{\sigma\sqrt{2\pi}}e^{-\frac{\delta^2}{2\sigma^2}}d\delta=1 \]
	由此,误差落在区间$ [-\delta,+\delta] $之间的概率为:\[ p=\int_{-\delta}^{+\delta}f(\delta)d\delta=\int_{-\delta}^{+\delta}\frac{1}{\sigma\sqrt{2\pi}}e^{-\frac{\delta^2}{2\sigma^2}}d\delta \]
	将上式进行变量转换,设\[ t=\frac{\delta}{\sigma},dt=\frac{d\delta}{\sigma} \]
	经变换,上式成为\[ p=\frac{1}{\sqrt{2\pi}}\int_{-t}^{+t}e^{-\frac{t^2}{2}}dt=\frac{2}{\sqrt{2\pi}}\int_{0}^{+t}e^{-\frac{t^2}{2}}dt=2\Phi(t) \]
	\[ \Phi(t)=\frac{1}{\sqrt{2\pi}}\int_{0}^{+t}e^{-\frac{t^2}{2}}dt \]
	若某随机误差在$ \pm t\sigma $范围内出现的概率为$ 2\Phi(t) $,则超出的概率为\[ \alpha=1-2\Phi(t) \]
	不同的$ t $时超出$ |\delta| $的概率是不同的,取不同的$ t $值时,极限误差可用下式表示:\[ \delta_{lim}x=\pm t\delta \]
	\textbf{算术平均值的极限误差} \[ \delta_{lim}\bar{x}=\pm t_a\sigma_{\bar{x}} \]
	式中:$ t_a $为置信系数;$ \alpha $为超出极限误差的概率;$ \sigma_{\bar{x}} $为$ n $次测量的算术平均值标准差。
	\item \textbf{权}
	
	\qquad 在等精度测量中,各个测得的值我们认为是同样可靠的,用所有值的算术平均值作为最后的测量结果。但是在不等精度测量中,各个测量值的可靠程度是不一样的,所以我们用权来说明测量的可靠程度。我们根据算术平均值标准差以及测量的次数来确定权,假设同一测量量有$ m $组不等精度的测量结果,可表示为\[ p_1:p_2:...:p_m=\frac{1}{\sigma_{\bar{x_1}}^2}:\frac{1}{\sigma_{\bar{x_2}}^2}:...:\frac{1}{\sigma_{\bar{x_m}}^2} \]
	\item \textbf{加权算术平均值}
	
	\qquad 在对m组测量量进行不等精度的测量时,得到的结果是$ \bar{x_1},\bar{x_2},...,\bar{x_m} $,设相应的测量次数为$ n_1,n_2,...,n_m $,可以得出全部测量的算术平均值为\[ \bar{x}=\frac{\sum\limits_{i=1}^{m}p_i\bar{x}_i}{\sum\limits_{i=1}^{m}p_i} \]
	\item \textbf{加权算术平均值的标准差}\[ \bar{s}=\sqrt{\frac{\sum\limits_{i=1}^{m}p_iv_{\bar{x_i}}^2}{(m-1)\sum\limits_{i=1}^{m}p_i}} \]
\end{enumerate}
\subsubsection{系统误差}
\begin{enumerate}
	\item \textbf{系统误差的产生原因}
	
	\qquad 系统误差是由固定不变的或者是按照确定的规律变化的因素所造成的。它是由多方面因素引起的:测量装置方面,环境方面的因素,测量方法的因素,测量人员方面的因素\footnote{《误差理论与数据处理》第41页,详情见参考文献[1]}。
	\item \textbf{系统误差的发现}
	
	\qquad 发现系统误差的方法有很多种,在组内有,实验对比法、残余误差观察法、残余误差校核法、不同公式计算标准差法;在组间有,计算数据比较法、秩和校验法、$ t $检验法。
	\item \textbf{系统误差的减小与排除}
	
	\qquad 我们要消除系统误差有两种方法,第一种是从产生误差根源上消除,第二种是用修正方法消除系统误差。
	
	\qquad 常用的不变系统误差消除法有:代替法、抵消法、交换法。线性系统误差消除法有:对称法。周期性系统误差消除法有:半周期法。
\end{enumerate}
\subsubsection{粗大误差}
在一系列重复的测量数据中,如果有个别的数据(最大值或最小值)严重地偏离了它所属样本的其他数据,则可以怀疑该组数据中含有粗大误差。
\begin{enumerate}
	\item \textbf{粗大误差产生的原因}
	
	\qquad 产生粗大误差的原因是多方面的,不过大致可以归纳为测量人员的主观原因和客观外界条件的原因。
	\item \textbf{粗大误差的特点}
	
	\qquad 粗大误差的特点可以总结为以下三点,第一,在引起粗大误差的各种因素中,很难事先预测到,所以它具有突发性。第二,产生的粗大误差一般都远大于正常测量值之间的距离,所以它一般为较大的数值。第三,含有粗大误差的异常数据个数一般都很少,所以它有数量少的特点。
	\item \textbf{判别粗大误差的准则}
	
	\qquad 遇到粗大误差时,我们要慎重对待,并要根据判别准则予以确定。通常用来判别粗大误差的准则有:$ 3\sigma $准则(莱以特准则)、罗曼诺夫斯基准则($ t $检验准则)、格拉布斯准则、狄克逊准则、奈尔准则、精细准则、肖维涅准则。
	
	\qquad 我们主要介绍狄克逊准则,因为它不需要事先求出标准差$ \sigma $,而且我们后面用到该方法检验粗大误差。狄克逊研究了$ x_1,x_2,...,x_n $的顺序统计量$ x_{(i)} $的分布,当$ x_{(i)} $服从正态分布时,得到$ x_{(i)} $的统计量
	\[ \begin{cases}
		r_{10}=\frac{x_{(n)}-x_{(n-1)}}{x_{(n)}-x_{(1)}}\\
		r_{11}=\frac{x_{(n)}-x_{(n-1)}}{x_{(n)}-x_{(2)}}\\
		r_{21}=\frac{x_{(n)}-x_{(n-2)}}{x_{(n)}-x_{(2)}}\\
		r_{22}=\frac{x_{(n)}-x_{(n-2)}}{x_{(n)}-x_{(3)}}
	\end{cases} \]
	的分布,选定显著度$ \alpha $,得到统计量的临界值$ r_0(n,\alpha) $,当测量值的统计值$ r_{ij} $大于临界值,则认为$ x_n $含有粗大误差。
	
	\qquad 统计量的临界值$ r_0(n,\alpha) $的查询见附录1 狄克逊检验统计量和临界值。
\end{enumerate}

\subsubsection{等精度测量数据的误差分析}
前面我们讨论了三类测量误差,它们的特点各异,因而处理的方法也有较大差别。等精度测量处理过程如图1所示,我们举一个等精度测量处理过程的例子。
\paragraph{例 } 对某一轴径进行等精度测量9次,得到表1的数据(单位略),求测量结果。

\begin{figure}[H]
	\centering
	\begin{tabular}{|p{3cm}<{\centering}|p{3cm}<{\centering}|p{3cm}<{\centering}|p{3cm}<{\centering}|}
	\hline 
		序号	&  $ l $&  $ v_i $&  $ v_i^2 $\\	\hline 
		1&  24.774&  -0.001&  0.000001\\ \hline 
		2&  24.778&  +0.003&  0.000009\\ \hline 
		3&  24.771&  -0.004&  0.000016\\ \hline
		4&  24.780&  +0.005&  0.000025\\ \hline
		5&  24.772&  -0.003&  0.000009\\ \hline
		6&  24.777&  +0.002&  0.000004\\ \hline
		7&  24.773&  -0.002&  0.000004\\ \hline
		8&  24.775&		  0&  		 0\\ \hline
		9&  24.774&  -0.001&  0.000001\\ \hline
		\multicolumn{4}{|c|}{$ \sum\limits_{i=1}^{9}x_i=222.974,\quad\bar{x}=24.775,\quad\sum\limits_{i=1}^{9}v_i=-0.001,\quad\sum\limits_{i=1}^{9}v_i^2 =0.000069$}\\ \hline
	\end{tabular}
	\captionsetup{type=table}
	\caption{某9次等精度测量数据}
\end{figure}
\begin{figure}[H]
	\centering
	\begin{tikzpicture}[
	>=latex,
	node distance=5mm,
	hv path/.style={to path={-| (\tikztotarget)}},
	vh path/.style={to path={|- (\tikztotarget)}},
	startend/.style={draw,rectangle,rounded corners=2mm,minimum size = 6mm,	thick},
	ioput/.style={draw,trapezium,trapezium left angle=60, trapezium right angle=120,inner sep = 5pt},
	chuli/.style={draw,rectangle,minimum size=6mm,thick},
	panduan/.style={draw,diamond,minimum size=6mm,shape aspect=3,inner sep = 0.1pt,thick}
	]
	\node	(a)		[startend]				{开始};
	\node	(b)		[ioput,below=of a]		{导入测量数据$ l_i(i=1,2,...,n) $};
	\node	(c)		[panduan,below=of b]	{是否有粗大误差?};
	\node	(d)		[chuli,right=of c]		{剔除含有粗大误差数据};
	\node	(e)		[panduan,below=of c]	{是否有系统误差?};
	\node	(f)		[chuli,below=of e]		{计算算术平均值$ \bar{x} $};
	\node	(g)		[chuli,below=of f]		{修正算术平均值$ \bar{x}=\bar{x}+\Delta $};
	\node	(h)		[chuli,below=of g]		{计算单次测量的标准差$ s $};
	\node	(i)		[chuli,below=of h]		{计算算术平均值标准差$ s(\bar{x}) $};
	\node	(j)		[chuli,below=of i]		{根据显著性水平及分布查表得$ t_\alpha $};
	\node	(k)		[chuli,below=of j]		{计算算术平均值极限误差$ \delta_{lim}\bar{x}=\pm t_\alpha s(\bar{x}) $};
	\node	(l)		[chuli,below=of k]		{给出测量结果$ L=\bar{x}+\delta_{lim}\bar{x} $};
	\node	(m)		[startend,below=of l]	{结束};
	
	\draw[->](a)--(b);
	\draw[->](b)--(c);
	\draw[->](c)--(d);
	\path (d.north) edge [->,vh path]($(b.south)!.5!(c.north)$);
	\draw[->](c)--(e);
	\draw[->](e)--(f);
	\draw[->](f)--(g);
	\draw[->](g)--(h);
	\draw[->](h)--(i);
	\draw[->](i)--(j);
	\draw[->](j)--(k);
	\draw[->](k)--(l);
	\draw[->](l)--(m);
	
	\node at (0.25,-3.6){否};
	\node at (2.15,-2.5){是};
	\end{tikzpicture}
	\caption{等精度测量数据处理流程图}
\end{figure}
\begin{enumerate}
	\item \textbf{粗大误差判别(用狄克逊准则差别)}
	
	\qquad 将数据进行从小到大排序,可以得到最小值$ x_{(1)} $和最大值$ x_{(9)} $。\[ x_{(1)}=24.771,x{(9)}=24.780 \]
	首先判断最大值$ x_{(9)} $,因$ n=9 $,帮计算得统计量$ r_{11} $为\[ r_{11}=\frac{x_{(9)}-x_{(8)}}{x_{(9)}-x_{(2)}}=\frac{24.780-24.778}{24.780-24.772}=\frac{0.002}{0.008}=0.400 \]
	查狄克逊准则临界表\footnote{见附录1}可知\[ r_0(9,0.05)=0.512 \]\[ r_{11}=0.4<r_0(9,0.05)=0.512 \]
	可以判断最大值$ x_{(9)} $不含有粗大误差,再对最小值$ x_{(1)} $计算相应统计量$ r'_{11} $
	\[ r'_{11}=\frac{x_{(1)}-x_{(2)}}{x_{(1)}-x_{(8)}}=\frac{24.771-24.772}{24.771-24.778}=\frac{-0.001}{-0.005}=0.167 \]
	\[ r'_{11}=0.167<r_0(9,0.05)=0.512 \]
	可以判断最小值$ x_{(1)} $不含有粗大误差,故可以判断该测量数据不含有粗大误差。
	\item \textbf{系统误差差别(用残余误差观察法)}
	
	\qquad 如图2所示,发现残余误差大体相同,无明显变化规律,可以认为不存在系统误差。
	\begin{figure}[H]
		\centering
		\begin{tikzpicture}
			\begin{axis}[ymajorgrids,minor tick num=1,width=15cm,height=7cm]
				\addplot[only marks] coordinates {
					(1,-0.001)
					(2,0.003)
					(3,-0.004)
					(4,0.005)
					(5,-0.003)
					(6,0.002)
					(7,-0.002)
					(8,0)
					(9,-0.001)
				};
			\end{axis}
		\end{tikzpicture}
		\caption{残余误差分布}
	\end{figure}

	\item \textbf{随机误差处理}
		\begin{enumerate}
			\item 求算术平均值:\[ \bar{x}=\frac{\sum\limits_{i=1}^{n}l_i}{n}=\frac{222.974}{9}mm=24.7749mm\approx24.775mm \]
			\item 求残余误差(见列表):\[ v_i=l_i-\bar{x} \]
			\item 校核算术平均值和残余误差。\[ \left| \sum\limits_{i=1}^{9}v_i \right|=0.001mm<\left(\frac{n}{2}-0.5\right)A=4\times0.001mm=0.004mm  \]
			以上计算正确,否则,应重新进行计算和校核:
			\item 求单次测量的标准差(贝塞尔公式)\[ s=\sqrt{\frac{\sum\limits_{i=1}^{n}v_i^2}{n-1}}=\sqrt{\frac{0.000069}{8}mm^2}=0.0029mm \]
			\item 求算术平均值的标准差:\[ \bar{s}=\frac{s}{\sqrt{n}}\approx0.001mm \]
			\item 求算术平均值的极限误差。
			
			\qquad 因为测量的次数较少,算术平均值的极限误差按$ t $分布计算,已知$ v=n-1=8 $,取$ \alpha=0.05 $查得$ t $分布临界值表得$ t_\alpha=2.31 $,则算术平均值的极限误差$ \delta_{lim}\bar{x} $为\[ \delta_{lim}\bar{x}=\pm t_\alpha\sigma_{\bar{x}}=\pm2.31\times0.001mm=\pm0.0023mm \]
		\end{enumerate}
	\item \textbf{测量结果}\[ L=\bar{x}\pm\delta_{lim}\bar{x}=(24.755\pm0.0023)mm \]
\end{enumerate}
\subsubsection{不等精度测量数据的误差分析}
\begin{figure}[H]
	\centering
	\begin{tikzpicture}[
	>=latex,
	node distance=5mm,
	hv path/.style={to path={-| (\tikztotarget)}},
	vh path/.style={to path={|- (\tikztotarget)}},
	startend/.style={draw,rectangle,rounded corners=2mm,minimum size = 6mm,	thick},
	ioput/.style={draw,trapezium,trapezium left angle=60, trapezium right angle=120,inner sep = 5pt},
	chuli/.style={draw,rectangle,minimum size=6mm,thick},
	panduan/.style={draw,diamond,minimum size=6mm,shape aspect=3,inner sep = 0.1pt,thick}
	]
	\node	(a)		[startend]				{开始};
	\node	(b)		[ioput,below=of a]		{导入测量数据$ L_i(i=1,2,...,m) $};
	\node at (-4,-2.4)	(cl)	[chuli]		{计算算术平均值$ \bar{x_1} $};
	\node at (4,-2.4)	(cr)	[chuli]		{计算算术平均值$ \bar{x_m} $};
	\node	(dl)	[panduan,below=of cl]	{是否有粗大误差?};
	\node	(dr)	[panduan,below=of cr]	{是否有粗大误差?};
	\node	(el)	[chuli,right=of dl,align=center]		{剔除含有粗\\大误差数据};
	\node	(er)	[chuli,right=of dr,align=center]		{剔除含有粗\\大误差数据};
	\node	(fl)	[panduan,below=of dl]	{是否有系统误差?};
	\node	(fr)	[panduan,below=of dr]	{是否有系统误差?};
	\node	(gl)	[chuli,below=of fl]		{修正算术平均值$ \bar{x_1}=\bar{x_1}+\Delta_1 $};
	\node	(gr)	[chuli,below=of fr]		{修正算术平均值$ \bar{x_m}=\bar{x_m}+\Delta_m $};
	\node	(hl)	[chuli,below=of gl]		{计算单次测量的标准差$ s_1 $};
	\node	(hr)	[chuli,below=of gr]		{计算单次测量的标准差$ s_m $};
	\node	(il)	[chuli,below=of hl]		{计算加权算术平均值标准差$ s(\bar{x_1}) $};
	\node	(ir)	[chuli,below=of hr]		{计算加权算术平均值标准差$ s(\bar{x_m}) $};
	\node	(jl)	[chuli,below=of il]		{计算相应的权$ p_1=1/(s(\bar{x_1}))^2 $};
	\node	(jr)	[chuli,below=of ir]		{计算相应的权$ p_m=1/(s(\bar{x_m}))^2 $};
	\node at (0,-12)	(k)		[chuli]		{计算加权算术平均值$ \bar{x}=\frac{\sum\limits_{i=1}^{m}p_i\bar{x_i}}{\sum\limits_{i=1}^{m}p_i} $};
	\node	(l)		[chuli,below=of k]		{计算加权算术平均值标准差$ \bar{s}=\bar{s_i}\sqrt{\frac{p_i}{\sum\limits_{i=1}^{m}p_i}} $};
	\node	(m)		[chuli,below=of l]		{计算加权算术平均值极限误差$ \delta_{lim}\bar{x}=\pm t_\alpha s(\bar{x}) $};
	\node	(n)		[chuli,below=of m]		{给出测量结果$ L=\bar{x}+\delta_{lim}\bar{x} $};
	\node	(o)		[startend,below=of n]	{结束};

	\draw[->](a)--(b);
	\path (b) edge [->,hv path](cl);
	\path (b) edge [->,hv path](cr);
	\draw[->](cl)--(dl);\draw[->](cr)--(dr);
	\draw[->](dl)--(el);\draw[->](dr)--(er);
	\path (el) edge [->,vh path](cl);
	\path (er) edge [->,vh path](cr);
	\draw[->](dl)--(fl);\draw[->](dr)--(fr);
	\draw[->](fl)--(gl);\draw[->](fr)--(gr);
	\draw[->](gl)--(hl);\draw[->](gr)--(hr);
	\draw[->](hl)--(il);\draw[->](hr)--(ir);
	\draw[->](il)--(jl);\draw[->](ir)--(jr);
	\path (jl) edge [->,vh path](k);
	\path (jr) edge [->,vh path](k);
	\draw[->](k)--(l);
	\draw[->](l)--(m);
	\draw[->](m)--(n);
	\draw[->](n)--(o);

	\node at (-1.9,-3.6){是};
	\node at (6.1,-3.6){是};
	\node at (-3.8,-4.8){否};
	\node at (4.2,-4.8){否};
	\node at (-3.8,-6.6){是};
	\node at (4.2,-6.6){是};
	\end{tikzpicture}
	\caption{不等精度测量数据处理流程图}
\end{figure}
\paragraph{例}对某一角度进行6组不等精度的测量,假设不含有随机误差和系统误差,各组测量的数据如下:

测6次得$ \alpha_1=75^\circ18'16'' $,测30次得$ \alpha_2=75^\circ18'10'' $,测24次得$ \alpha_3=75^\circ18'08'' $,测12次得$ \alpha_4=75^\circ18'16'' $,测12次得$ \alpha_5=75^\circ18'13'' $,测36次得$ \alpha_6=75^\circ18'09'' $,求得最后的测量结果。
\begin{enumerate}
	\item 根据测量的数据求出各组的权,有\[ p_1:p_2:p_3:p_4:p_5:p_6=1:5:4:2:2:6 \]取\[ p_1=1,p_2=5,p_3=4,p_4=2,p_5=2,p_6=6 \]
	\item 求出加权算术平均值$ \bar{\alpha} $:\[ \bar{\alpha}=\sum_{i=1}^{6}p_i\alpha_i=75^\circ18'10'' \]
	\item 求残余误差,由公式$ v_i=\alpha_i-\bar{\alpha} $得\[ v_1=-4'',v_2=0,v_3=-2'',v_4=6'',v_5=3'',v_6=-1'' \]
	\item 校核算术平均值及残余误差。用加权残余误差代数和等于0来校核。因:\[ \sum_{i=1}^{6}p_iv_i=0 \]所以计算正确。
	\item 求加权算术平均值的标准差:\[ \bar{s}=\sqrt{\frac{\sum\limits_{i=1}^{6}p_iv_i^2}{(m-1)\sum\limits_{i=1}^{6}p_i}}=1.1'' \]
	\item 求加权算术平均值的极限误差,我们可以根据测量数据得出,该数据服从正态分布,取置信系数$ t=3 $,则极限误差为\[ \delta_{lim}\bar{\alpha}=\pm3\bar{s}=\pm3.3'' \]
	\item 测量结果为\[ \alpha=\bar{\alpha}+\delta_{lim}\bar{\alpha}=75^\circ18'10''\pm3.3'' \]
\end{enumerate}
\subsection{误差的合成}
前面我们说讨论的都是直接测量的误差计算,可是在许多情况下,并不能对被测对象进行直接地测量,或者直接测量不能满足精度上的需要,需要采用间接测量。间接测量误差是各个直接测量值误差的函数,称这种误差为函数误差\footnote{《误差理论与数据处理》第79页函数误差的定义,详情见参考文献[1]}。其实就是研究误差的传递问题,对于这种有确定关系的误差计算,也中叫做误差的合成。

间接测量的数学模型为\[ y=f(x_1,x_2,...,x_n) \]
式中:$ x_1,x_2,...,x_n $为直接测量值,$ y $为间接测量值。

函数$ y $的全微分表达式为\[ dy=\frac{\partial f}{\partial x_1}dx_1+\frac{\partial f}{\partial x_2}dx_2+...+\frac{\partial f}{\partial x_n}dx_n \]
\subsubsection{系统误差的合成}
已定系统误差的合成,已定系统误差是指误差大小和方向均已确定的系统误差。在测量过程中,若有$ r $个已定系统误差,其误差分别为$ \Delta_1,\Delta_2,...,\Delta_r $,相应的误差传递函数为$ a_1,a_2,...,a_r $,则按代数和进行合成,求得总的系统误差为\[ \Delta=\sum_{i=1}^{r}a_i\Delta i \]

未定系统误差的合成,采用随机误差的合成公式。
\begin{enumerate}
	\item \textbf{标准差的合成}
	
	\qquad 当测量的过程中有$ s $个单项的未定系统误差,标准差分别为$ u_1,u_2,...,u_s $,相应的误差传递系数为$ a_1,a_2,...,a_s $,则合成后未定系统误差的总标准差为\[ u=\sqrt{\sum_{i=1}^{s}(a_iu_i)^2+2\sum_{1\leq i\leq j}^{s}\rho_{ij}a_ia_ju_iu_j} \]
	当$ \rho_{ij}=0 $时,则有\[ u=\sqrt{\sum_{i=1}^{s}(a_iu_i)^2} \]
	\item \textbf{极限误差的合成}
	
	\qquad 因为极限误差为$ e_i=\pm t_iu_i,i=1,2,...,s $,总的极限误差为$ e=\pm tu $,则可得\[ e=\pm t\sqrt{\sum_{i=1}^{s}(a_iu_i)^2+2\sum_{1\leq i\leq j}^{s}\rho_{ij}a_ia_ju_iu_j} \]
	当各个单项未定系统误差均服从正态分布,且$ \rho_{ij}=0 $时,则简化为\[ e=\pm\sqrt{\sum_{i=1}^{s}(a_ie_i)^2} \]
\end{enumerate}
\subsubsection{随机误差的合成}
\begin{enumerate}
	\item \textbf{标准差的合成}
	
	\qquad 若有$ q $个单项随机误差,它们的标准差分别为$ \sigma_1,\sigma_2,...,\sigma_q $其相应的误差传递函数为$ a_1,a_2,...,a_q $。合成的总标准差为\[ \sigma=\sqrt{\sum_{i=1}^{q}(a_i\sigma_i)^2+2\sum_{1\leq i<j}^{q}\rho_{ij}a_ia_j\sigma_i\sigma_j} \]
	
	\qquad 当相关系数$ \rho_{ij}=0 $,则有\[ \sigma=\sqrt{\sum_{i=1}^{q}(a_i\sigma_i)^2} \]
	\item \textbf{极限误差的合成}
	
	\qquad  已知各单项极限误差为$ \delta_1,\delta_2,...,\delta_q $,且置信概率相同,则总极限误差为\[ \delta=\pm\sqrt{\sum_{i=1}^{q}(a_i\delta_i)^2+2\sum_{1\le i<j}^{q}\rho_{ij}a_ia_j\delta_i\delta_j} \]
	式中:$ a_i $为各极限误差传递系数;$ \rho_{ij} $为任意误差间的相关系数。
	
	\qquad 引入置信系数后得:\[ \delta=\pm t\sqrt{\sum_{i=1}^{q}(a_i\delta_i)^2+2\sum_{1\le i<j}^{q}\rho_{ij}a_ia_j\delta_i\delta_j} \]
\end{enumerate}

\subsection{测量不确定度}
\subsubsection{标准不确定度的A类评定}
当被测量$ Y $取决于它$ N $个量$ X_1,X_2,...,X_N $时,则$ Y $的估计值$ y $的标准不确守度将取决于$ X_i $的估计值$ x_i $的标准不确定度$ u_{x_i} $,为此要先评定$ x_i $的标准不确定度$ u_{x_i} $,在其他$ X_j(j\neq i) $保持不变的情况下,仅对$ X_i $进行$ n $次等精度独立测量,用统计法由$ n $个观测值求得单次测量标准差$ \sigma_i $,则$ x_i $的标准不确定度$ u_{x_i} $的数值为\[ u_{x_i}=\frac{\sigma_i}{\sqrt{n}} \]
\subsubsection{标准不确定度的B类评定}
\begin{enumerate}
	\item 当测量估计值$ x $为正态分布时,由所取置信概率$ P $的分布区间半宽$ a $与包含因子$ k_p $来估计标准不确定度,即\[ u_x=\frac{a}{k_p} \]
	式中包含因子$ k_p $的数值由正态分布积分表查得。
	\item 当估计值$ x $取自相关资料,所给出的测量不确定度$ U_x $为标准差的$ k $倍时,则标准不确定度为\[ u_x=\frac{U_x}{k} \]
	\item 若已知估计值$ x $落在区间$ (x-a,x+a) $内的概率为1,且$ x $服从均匀分布,标准不确定度为\[ u_x=\frac{a}{\sqrt{3}} \]
	\item 当$ x $服从在区间$ (x-a,x+a) $内的三角分布,标准不确定度为\[ u_x=\frac{a}{\sqrt{6}} \]
	\item 当$ x $服从在区间$ (x-a,x+a) $内的反正弦分布,标准不确定度为\[ u_x=\frac{a}{\sqrt{2}} \]
\end{enumerate}
\subsubsection{自由度的确定}
\begin{enumerate}
	\item 对于标准不确定度A类评定的自由度,如果用贝塞尔计算的标准差,其自由度为\[ v=n-1 \],而用其他方法计算的标准差,自由度有所不同。
	\item 对于标准不确定度B类评定的自由度,由估计$ u $的相对标准差来确定,定义为\[ v=\frac{1}{2(\frac{\sigma_u}{u})^2} \]
	式中:$ \delta_u $为评定$ u $的标准差,$ \sigma_u/u $为评定$ u $的相对标准差。
\end{enumerate}
\subsubsection{测量不确定度合成}
\begin{enumerate}
	\item 合成标准不确定度\[ u_c=\sqrt{\sum_{i=1}^{n}u_i^2+2\sum_{1\le i<j}^{N}\rho_{ij}u_iu_j} \]
	其测量结果表示为\[ Y=y\pm u_c \]
	\item 展伸不确定度,它由合成标准不确定度$ u_c $乘以包含因子$ k $得到,即\[ U=ku_c \]
	其测量结果为\[ Y=y\pm U \]
\end{enumerate}
当各不确定度分量$ u_i $相互独立时,合成标准不确定度的自由度$ v $为\[ v=\frac{u_c^4}{\sum\limits_{i=1}^{N}\frac{u_i^4}{v_i}} \]
\subsection{最小二乘法处理}
测量结果的最可信赖值应使残余误差平方和最小(不等精度测量时,使加权残余误差平方和最小),这就是最小二乘法原理\footnote{《
	误差理论与数据处理》第125页最小二乘法的定义,详情见参考文献[1]}。
\subsubsection{等精度测量线性参数最小二乘法处理的正规方程}
\begin{figure}[H]
	\centering
	\begin{tikzpicture}[
	>=latex,
	node distance=5mm,
	chuli/.style={draw,rectangle,minimum size=6mm,thick}
	]

	\node	(a)	[chuli]								{根据具体问题列出误差方程};
	\node	(b)	[chuli,below=of a,align=center]		{按最小二乘法原理,利用函数求极值\\的方法将误差方程转换为正规方程};
	\node	(c)	[chuli,below=of b]					{求解正规方程,得到待求的估计量};
	\node	(d)	[chuli,below=of c]					{给出估计量的不确定度评定};
	
	\draw[->](a)--(b);
	\draw[->](b)--(c);
	\draw[->](c)--(d);

	\end{tikzpicture}
	\caption{线性参数最小二乘法处理流程图}
\end{figure}
线性参数的误差方程式为
\[ \begin{cases}
	v_1 = l_1-(a_{11}x_1+a_{12}x_2+...+a_{1t}x_t)\\
	v_2 = l_2-(a_{21}x_1+a_{22}x_2+...+a_{2t}x_t)\\
	\qquad \vdots\\
	v_n = l_n-(a_{n1}x_1+a_{n2}x_2+...+a_{nt}x_t)\\
\end{cases} \]
在等精度测量中,最小二乘条件式为\[ \sum_{i=1}^{n}v_i^2=v_1^2+v_2^2+...+v_n^2=min \]
经过转换和简化正规方程可写成
\[ \begin{cases}
	a_{11}v_1+a_{21}v_2+...+a_{n1}v_n=0\\
	a_{12}v_1+a_{22}v_2+...+a_{n2}v_n=0\\
	\qquad \vdots\\
	a_{1t}v_1+a_{2t}v_2+...+a_{nt}v_n=0
\end{cases} \]
转换为矩阵形式:
\[ \begin{bmatrix}
	a_{11}&a_{21}&\dots&a_{n1}\\
	a_{12}&a_{22}&\dots&a_{n2}\\
		  & 	 &\vdots\\
	a_{1t}&a_{2t}&\dots&a_{nt}
\end{bmatrix}\begin{bmatrix}
	v_1\\v_2\\\vdots\\v_n
\end{bmatrix}=\begin{bmatrix}
	0\\0\\\vdots\\0
\end{bmatrix} \]
\subsubsection{不等精度测量线性参数最小二乘法处理}
\subsection{回归分析}
\subsubsection{一元线性回归}
\subsubsection{一元非线性回归}
\subsubsection{多元线性回归}


















