\usepackage{graphicx}
\usepackage{float}
\usepackage{caption}
\usepackage{amsmath}
\usepackage{multirow}
\usepackage{booktabs}
\usepackage{array}
\usepackage{listings}
\usepackage{color}
\usepackage{pgfplots}
\usepackage{pstricks}
\usepackage{tikz}
\usetikzlibrary{arrows,decorations.pathmorphing,backgrounds,positioning,fit,petri,shapes,snakes,calc}
\usepackage{fontspec}
\setmainfont{Times New Roman}

\definecolor{codegreen}{rgb}{0,0.6,0}
\definecolor{codegray}{rgb}{0.5,0.5,0.5}
\definecolor{codepurple}{rgb}{0.58,0,0.82}
\definecolor{backcolour}{rgb}{0.95,0.95,0.92}

%设计页面尺寸
\usepackage{geometry}
\geometry{a4paper,left=2.5cm,right=2cm,top=2cm,bottom=2cm}
\usepackage{fancyhdr}
\pagestyle{fancy}
\fancyhf{}
\cfoot{\thepage}
\fancyhead[C]{防灾科技学院毕业设计}

%设置目录格式
\usepackage{tocloft}
\setlength\cftbeforepartskip{0.3em}
\setlength\cftbeforesecskip{0.3em}
\setlength\cftbeforesubsecskip{0.3em}
\setlength\cftbeforesubsubsecskip{0.3em}
\renewcommand\cftpartleader{\cftdotfill{\cftpartdotsep}}
\renewcommand\cftpartdotsep{\cftdotsep}
\renewcommand\cftsecleader{\cftdotfill{\cftsecdotsep}}
\renewcommand\cftsecdotsep{\cftdotsep}
\renewcommand\cftpartfont{\bfseries\zihao{4}}
\renewcommand\cftsecfont{\bfseries\zihao{4}}

%设计章节格式
\CTEXsetup[format={\bfseries\zihao{4}}]{paragraph}
\CTEXsetup[format={\bfseries\zihao{4}},indent={0pc},beforeskip={0.3em},afterskip={0.3em}]{part}
\CTEXsetup[format={\bfseries\zihao{4}},indent={0pc},beforeskip={0.3em},afterskip={0.3em}]{section}
\CTEXsetup[format={\bfseries\zihao{-4}},indent={0pc},beforeskip={0.3em},afterskip={0.3em}]{subsection}
\CTEXsetup[format={\bfseries\zihao{-4}},indent={0pc},beforeskip={0.3em},afterskip={0.3em}]{subsubsection}

%书签和文档信息设置
\hypersetup{linkcolor=black,anchorcolor=black,filecolor=black,urlcolor=black,citecolor=black,bookmarksopen=true,pdftitle={基于GUI的误差理论与数据处理},pdfauthor={艾江南},pdfsubject={防灾科技学院毕业论文}}

%代码高亮设计
\lstset{columns=flexible,
	language=Matlab,
%	escapechar=',
	backgroundcolor=\color{backcolour},
	keywordstyle=\color{magenta},
%	numberstyle=\small\color{codegray},
	stringstyle=\color{codepurple},
	commentstyle=\color{codegreen},
	basicstyle=\small,
	breakatwhitespace=false,
	breaklines=true,
	captionpos=b,
	keepspaces=true,
%	numbers=left,
%	numbersep=8pt,
	showspaces=false,
	showstringspaces=false,
	showtabs=false,
	tabsize=4}

%定制命令
\newcommand\scite[1]{\textsuperscript{\cite{#1}}}
\newcommand{\tabincell}[2]{\begin{tabular}{@{}#1@{}}#2\end{tabular}}