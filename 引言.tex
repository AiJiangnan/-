\part*{引言}
\addcontentsline{toc}{part}{引言}
利用Matlab辅助教学手段,传统误差数据处理要花费很长时间才能完成的处理仅仅几秒钟就可以得出其结果,且非常直观。这在实际误差数据处理中具有很高的实用价值,可以节约大量的时间,达到事半功倍的效果。采用Matlab中的GUI完成误差理论与数据处理中误差分类以及有效数字与数据运算、算数平均值、标准差、测量不确定度、误差的合成、最小二乘法拟合和回归分析,能够快速准确地完成对数据的误差分析与处理。

另外,随着信息时代的到来,计算机已经渗入到各行各业中,通过计算机辅助设计、辅助计算都可以极大地提高效率。作为具有科学计算、符号运算、图形处理和人机交互界面设计等多种功能的软件工具,Matlab已经得到了业界的普遍认可,它功能强、应用广,在数学教学和数据处理中,更是如此。同时,Matlab的图形用户界面GUI(Graphical User Interface)功能让一些常用的数据处理方法设计成一个个运行在Matlab上的数据处理软件,这不仅对于用户在视觉上更容易接受,还大大增加了Matlab程序设计的重用性。

设计基于GUI的误差理论与数据处理系统的意义在于:

(1)正确理解和认识误差的性质,分析误差产生的原因,从根本上消除和减小误差;

(2)正确处理测量和实验数据,以得到更接近于真值的数据;

(3)正确评定测量结果,合理评定测量结果的可靠性;

(4)应用数据处理软件,高效解决实际问题。