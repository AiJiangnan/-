\part*{引言}
\addcontentsline{toc}{part}{引言}
人们在进行科学研究、经济建设以及学术研究等很多情况下,任何科学实验和工程实践都离不开测量这一环节,由于在测量结果中多多少少都会存在误差,影响测量的实验数据的可信赖性,甚至失去测量数据的科学价值和实用意义。因此,对于测量数据中误差的影响,以及对它的减小和控制就变得重要了。所以,在长期的实践中,人们也越来越认识到误差理论的重要性,特别是在当今信息时代,大量的数据信息,必须经过合理的数据处理并给出科学的评价,才有实际价值\scite{1}。

计算机的高速发展使得复杂的计算变得简单而又快捷,很多数学应用软件成为复杂误差数据处理的重要工具。利用Matlab辅助教学手段,传统误差数据处理要花费很长时间才能完成的处理仅仅几秒钟就可以得出其结果,且非常直观。这在实际误差数据处理中具有很高的实用价值,可以节约大量的时间,达到事半功倍的效果。采用Matlab完成误差理论与数据处理中误差分类以及有效数字与数据运算、算数平均值、标准差、测量不确定度、误差的合成、最小二乘法拟合和回归分析,能够快速准确地完成对数据的误差分析与处理。

同时,Matlab语言在复杂的数值运算、矩阵运算等方面都具有明显的优势,所以在开发对大量数据进行处理的软件时,利用Matlab 的图形用户界面GUI(Graphical User Interface)是一个很好的选择,而现在利用Matlab GUI进行软件开发的例子却不多\scite{2}。Matlab GUI让一些常用的数据处理方法设计成一个运行在Matlab上的图形界面软件,这不仅对于用户在视觉上更容易接受,还大大增加了Matlab程序设计的重用性。
