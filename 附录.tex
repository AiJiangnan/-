\part*{附录}
\addcontentsline{toc}{part}{附录}
\begin{center}
	\textbf{附录1 狄克逊检验统计量和临界值}
\end{center}

\centering
\begin{tabular}{|p{4cm}<{\centering}|p{2cm}<{\centering}|p{2cm}<{\centering}|p{2cm}<{\centering}|p{2cm}<{\centering}|}
	\hline
	统计量&$ n $& 0.10&\tabincell{c}{$\alpha$\\0.05 }& 0.01 \\	\hline
	\multirow{5}*{\tabincell{l}{$ r_{10}=\frac{x_{(n)}-x_{(n-1)}}{x_{(n)}-x_{(1)}}$\\$r'_{10}=\frac{x_{(1)}-x_{(2)}}{x_{(1)}-x_{(n)}} $}}														&3	&0.886	&0.941	&0.988	\\	\cline{2-5}
	&4	&0.679	&0.765	&0.889	\\	\cline{2-5}
	&5	&0.557	&0.642	&0.780	\\	\cline{2-5}
	&6	&0.482	&0.560	&0.698	\\	\cline{2-5}
	&7	&0.434	&0.507	&0.637	\\	\hline
	\multirow{3}*{\tabincell{l}{$ r_{11}=\frac{x_{(n)}-x_{(n-1)}}{x_{(n)}-x_{(2)}} $\\$ r'_{11}=\frac{x_{(1)}-x_{(2)}}{x_{(1)}-x_{(n-1)}} $}}														&8	&0.479	&0.554	&0.683	\\	\cline{2-5}
	&9	&0.441	&0.512	&0.635	\\	\cline{2-5}
	&10	&0.409	&0.447	&0.597	\\	\hline
	\multirow{3}*{\tabincell{l}{$ r_{21}=\frac{x_{(n)}-x_{(n-2)}}{x_{(n)}-x_{(2)}} $\\$ r'_{21}=\frac{x_{(1)}-x_{(3)}}{x_{(1)}-x_{(n-1)}} $}}														&11	&0.517	&0.576	&0.670	\\	\cline{2-5}
	&12	&0.490	&0.546	&0.642	\\	\cline{2-5}
	&13	&0.467	&0.521	&0.615	\\	\hline
	\multirow{17}*{\tabincell{l}{$ r_{22}=\frac{x_{(n)}-x_{(n-2)}}{x_{(n)}-x_{(3)}} $\\$ r'_{22}=\frac{x_{(1)}-x_{(3)}}{x_{(1)}-x_{(n-2)}} $}}														&14	&0.492	&0.548	&0.641	\\	\cline{2-5}
	&15	&0.472	&0.525	&0.616	\\	\cline{2-5}
	&16	&0.454	&0.507	&0.595	\\	\cline{2-5}
	&17	&0.438	&0.490	&0.577	\\	\cline{2-5}
	&18	&0.424	&0.475	&0.561	\\	\cline{2-5}
	&19	&0.412	&0.462	&0.547	\\	\cline{2-5}
	&20	&0.401	&0.450	&0.535	\\	\cline{2-5}
	&21	&0.391	&0.440	&0.524	\\	\cline{2-5}
	&22	&0.382	&0.430	&0.514	\\	\cline{2-5}
	&23	&0.374	&0.421	&0.505	\\	\cline{2-5}
	&24	&0.367	&0.413	&0.497	\\	\cline{2-5}
	&25	&0.360	&0.406	&0.489	\\	\cline{2-5}
	&26	&0.354	&0.399	&0.486	\\	\cline{2-5}
	&27	&0.348	&0.393	&0.475	\\	\cline{2-5}
	&28	&0.342	&0.387	&0.469	\\	\cline{2-5}
	&29	&0.337	&0.381	&0.463	\\	\cline{2-5}
	&30	&0.332	&0.378	&0.457	\\	\hline
\end{tabular}
\newpage
\begin{center}
	\textbf{附录2 GUI设计函数常用属性\footnote{参考文献[2]第9章Matlab GUI的组成与结构第196页}}
\end{center}
\begin{tabular}{|p{3cm}|p{5cm}|p{5cm}|}
	\hline
	\multicolumn{1}{|c|}{\textbf{对象属性}}	&\multicolumn{1}{c}{\textbf{意义}}	&\multicolumn{1}{|c|}{\textbf{取值及含义}}\\	\hline
	BackgroundColor	&uicontrol背景色。3元素的RGB向量或Matlab预先定义的颜色名称	&默认的背景色是浅灰色		\\	\hline
	Callback		&Matlab回调串,当uicontrol激活时,回调串传给函数eval	   &初始值为空矩阵			\\	\hline
	Max				&属性Value的最大许可值									 &默认值为1				\\	\hline
	Min				&属性Value的最小许可值									 &默认值为0				\\	\hline
	Position		&位置向量[left bottom width height]					   	&					    \\	\hline
	String			&文本字符串,uicontrol显示的文本						    &					 \\	\hline
	Style			&定义uicontrol的类型					&text,edit,pushbutton...				\\	\hline
	Units			&设置属性值的单位										&pixels:屏幕像素		\\	\hline
	Value			&uicontrol的当前值										&文本对象和按钮不设置该属性	\\	\hline
	UserData		&用户指定的数据。可以是矩阵、字符串等						&						\\	\hline
\end{tabular}
\newpage
\begin{lstlisting}[language=Matlab]
% 等精度测量数据误差分析
%
% 输入参数:原始数据、置信系数
% 输出参数:剔除粗大误差后数据、剔除粗大误差后残余误差、平均值、...
%                 剔除粗大误差后平均值、标准差、剔除粗大误差后标准差、...
%                 算术平均值标准差、结果
%

function [data1,v1,a,a1,s,s1,s1_x,x] = data_process1(data,t_a)

a = mean(data);
s = std(data);
data1 = data;%以后加入剔除粗大误差函数

a1 = mean(data1);
s1 = std(data1);
n1 = length(data1);
v1 = 1:n1;
for i=1:n1
	v1(i) = data1(i)-a1;
end
s1_x = s1/sqrt(n1);
sigama = t_a*s1_x;
x = roundn([a1 sigama],-3);
\end{lstlisting}

