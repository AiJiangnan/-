\part*{附录}
\addcontentsline{toc}{part}{附录}
\begin{center}
	\textbf{附录1 狄克逊检验统计量和临界值}
\end{center}

\centering
\begin{tabular}{|p{4cm}<{\centering}|p{2cm}<{\centering}|p{2cm}<{\centering}|p{2cm}<{\centering}|p{2cm}<{\centering}|}
	\hline
	统计量&$ n $& 0.10&\tabincell{c}{$\alpha$\\0.05 }& 0.01 \\	\hline
	\multirow{5}*{\tabincell{l}{$ r_{10}=\frac{x_{(n)}-x_{(n-1)}}{x_{(n)}-x_{(1)}}$\\$r'_{10}=\frac{x_{(1)}-x_{(2)}}{x_{(1)}-x_{(n)}} $}}														&3	&0.886	&0.941	&0.988	\\	\cline{2-5}
	&4	&0.679	&0.765	&0.889	\\	\cline{2-5}
	&5	&0.557	&0.642	&0.780	\\	\cline{2-5}
	&6	&0.482	&0.560	&0.698	\\	\cline{2-5}
	&7	&0.434	&0.507	&0.637	\\	\hline
	\multirow{3}*{\tabincell{l}{$ r_{11}=\frac{x_{(n)}-x_{(n-1)}}{x_{(n)}-x_{(2)}} $\\$ r'_{11}=\frac{x_{(1)}-x_{(2)}}{x_{(1)}-x_{(n-1)}} $}}														&8	&0.479	&0.554	&0.683	\\	\cline{2-5}
	&9	&0.441	&0.512	&0.635	\\	\cline{2-5}
	&10	&0.409	&0.447	&0.597	\\	\hline
	\multirow{3}*{\tabincell{l}{$ r_{21}=\frac{x_{(n)}-x_{(n-2)}}{x_{(n)}-x_{(2)}} $\\$ r'_{21}=\frac{x_{(1)}-x_{(3)}}{x_{(1)}-x_{(n-1)}} $}}														&11	&0.517	&0.576	&0.670	\\	\cline{2-5}
	&12	&0.490	&0.546	&0.642	\\	\cline{2-5}
	&13	&0.467	&0.521	&0.615	\\	\hline
	\multirow{17}*{\tabincell{l}{$ r_{22}=\frac{x_{(n)}-x_{(n-2)}}{x_{(n)}-x_{(3)}} $\\$ r'_{22}=\frac{x_{(1)}-x_{(3)}}{x_{(1)}-x_{(n-2)}} $}}														&14	&0.492	&0.548	&0.641	\\	\cline{2-5}
	&15	&0.472	&0.525	&0.616	\\	\cline{2-5}
	&16	&0.454	&0.507	&0.595	\\	\cline{2-5}
	&17	&0.438	&0.490	&0.577	\\	\cline{2-5}
	&18	&0.424	&0.475	&0.561	\\	\cline{2-5}
	&19	&0.412	&0.462	&0.547	\\	\cline{2-5}
	&20	&0.401	&0.450	&0.535	\\	\cline{2-5}
	&21	&0.391	&0.440	&0.524	\\	\cline{2-5}
	&22	&0.382	&0.430	&0.514	\\	\cline{2-5}
	&23	&0.374	&0.421	&0.505	\\	\cline{2-5}
	&24	&0.367	&0.413	&0.497	\\	\cline{2-5}
	&25	&0.360	&0.406	&0.489	\\	\cline{2-5}
	&26	&0.354	&0.399	&0.486	\\	\cline{2-5}
	&27	&0.348	&0.393	&0.475	\\	\cline{2-5}
	&28	&0.342	&0.387	&0.469	\\	\cline{2-5}
	&29	&0.337	&0.381	&0.463	\\	\cline{2-5}
	&30	&0.332	&0.378	&0.457	\\	\hline
\end{tabular}
\newpage
\begin{center}
	\textbf{附录2 GUI设计函数常用属性\footnote{参考文献[2]第9章Matlab GUI的组成与结构第196页}}
\end{center}
\begin{tabular}{|p{3cm}|p{5cm}|p{5cm}|}
	\hline
	\multicolumn{1}{|c|}{\textbf{对象属性}}	&\multicolumn{1}{c}{\textbf{意义}}	&\multicolumn{1}{|c|}{\textbf{取值及含义}}\\	\hline
	BackgroundColor	&uicontrol背景色。3元素的RGB向量或Matlab预先定义的颜色名称	&默认的背景色是浅灰色		\\	\hline
	Callback		&Matlab回调串,当uicontrol激活时,回调串传给函数eval	   &初始值为空矩阵			\\	\hline
	Max				&属性Value的最大许可值									 &默认值为1				\\	\hline
	Min				&属性Value的最小许可值									 &默认值为0				\\	\hline
	Position		&位置向量[left bottom width height]					   	&					    \\	\hline
	String			&文本字符串,uicontrol显示的文本						    &					 \\	\hline
	Style			&定义uicontrol的类型					&text,edit,pushbutton...				\\	\hline
	Units			&设置属性值的单位										&pixels:屏幕像素		\\	\hline
	Value			&uicontrol的当前值										&文本对象和按钮不设置该属性	\\	\hline
	UserData		&用户指定的数据。可以是矩阵、字符串等						&						\\	\hline
	Tag				&文本串												&						\\	\hline
	CreateFcn		&对象创建时执行的回调函数								&						\\	\hline
	DeleteFcn		&对象删除时执行的回调函数								&						\\	\hline
	CloseRequestFcn		&窗口关闭时执行的回调函数							&						\\	\hline
	ButtonDownFcn		&对象鼠标单击时执行的回调函数						  &						\\	\hline
\end{tabular}
\newpage
\begin{center}
	\textbf{附录3 界面设计代码}
\end{center}
\begin{enumerate}
	\item 设置界面图标:icon.m
	\begin{lstlisting}
 % 更改界面左上角图标
 newIcon = javax.swing.ImageIcon('./image/logo.png');
 figFrame = get(gcf,'JavaFrame');
 figFrame.setFigureIcon(newIcon);\end{lstlisting}
	\item 界面返回:page\_exit.m
	\begin{lstlisting}
 %% 返回上一级菜单
 close(gcf);
 clear all
 clc\end{lstlisting}
 	\item 主界面:index.m
	\begin{lstlisting}
 function index
 clear
 clc
 
 % 添加子文件夹路径
 addpath(genpath(pwd));
 
 %% 创建主界面
 s = get(0,'ScreenSize');% 获取计算机屏幕分辨率
 x = s(3)*0.15;
 y = s(4)*0.26;
 hf = figure('Name','基于GUI的误差理论与数据处理系统',...
     'NumberTitle','off',...
     'Units','pixels',...
     'Position',[x,y,600,450],...
     'MenuBar','none',...
     'Color','White',...
     'CloseRequestFcn',@hexit,...
     'Resize','off');
 
 % 更改界面左上角图标
 icon;
 
 % 学校名称
 axes('Units','pixels',...
     'Position',[0,350,430,100],...
     'CreateFcn',@school_logo);
 
 % 校训
 uicontrol(hf,...
     'Units','pixels',...
     'Position',[300,320,230,25],...
     'Style','text',...
     'String','崇德博智,扶危定倾',...
     'FontName','楷体',...
     'FontSize',18,...
     'FontWeight','bold',...
     'ForegroundColor',[7,51,123]/255,...
     'BackgroundColor','White');
 
 % 学校标志性建筑
 axes('Units','pixels',...
     'Position',[0,80,600,233],...
     'CreateFcn','imshow(''image/school.jpg'')');
 
 %% 一级菜单按钮
 uicontrol(hf,...
     'Units','pixels',...
     'Position',[160,35,280,25],...
     'Style','text',...
     'String','《误差理论与数据处理》',...
     'FontName','楷体',...
     'FontSize',18,...
     'FontWeight','bold',...
     'ForegroundColor',[7,51,123]/255,...
     'BackgroundColor','White',...
     'enable','inactive',...
     'ButtonDownFcn','subpage');
 uicontrol(hf,...
     'Units','pixels',...
     'Position',[160,5,280,25],...
     'Style','text',...
     'String','Error Theory and Data Processing',...
     'FontSize',12,...
     'FontWeight','bold',...
     'ForegroundColor',[7,51,123]/255,...
     'BackgroundColor','White',...
     'enable','inactive',...
     'ButtonDownFcn','subpage');
 
 % 显示学校名称PNG图片函数
 function school_logo(a,b)
 [I,c,alpha] = imread('image/school_logo.png');
 h = imshow(I);
 set(h,'AlphaData',alpha);
 
 %% 主界面退出对话框
 function hexit(a,b)
 he = questdlg('你确定退出吗?','退出程序','是','否','否');
 if strcmp(he,'是')
     close;
     clear;
     clc;
 end;\end{lstlisting}
	\item 二级子界面:subpage.m
	\begin{lstlisting}
 function subpage
 clear
 clc
 
 %% 创建主界面
 s = get(0,'ScreenSize');% 获取计算机屏幕分辨率
 x = s(3)*0.15;
 y = s(4)*0.26;
 hf = figure('Name','基于GUI的误差理论与数据处理系统',...
     'NumberTitle','off',...
     'Units','pixels',...
     'Position',[x,y,710,450],...
     'MenuBar','none',...
     'Color','White',...
     'Resize','off');
 
 % 更改界面左上角图标
 icon;
 
 %创建文字项
 t = 1:6;
 menu_string = {'测量数据基本处理','误差的合成','测量不确定度','最小二乘法处理',' 回归分析','返回'};
 menu_position = [0,240,230,25
                  240,240,230,25
                  480,240,230,25
                  0,10,230,25
                  240,10,230,25
                  480,10,230,25];
 %设置文字项属性
 for i = 1:length(t)
     t(i) = uicontrol(hf,...
         'Style','text',...
         'String',menu_string(i),...
         'FontName','微软雅黑',...
         'FontSize',14,...
         'FontWeight','bold',...
         'enable','inactive',...
         'Units','pixels',...
         'Position',menu_position(i,:),...
         'ButtonDownFcn',strcat('subsubpage',num2str(i)));
 end
 
 %创建图片项
 p1 = axes('CreateFcn','bar([2,1,3,5,3]);');
 p2 = axes('CreateFcn','imshow(''image/photo_2.jpg'');');
 p3 = axes('CreateFcn','imshow(''image/photo_3.jpg'');');
 p4 = axes('CreateFcn',@hplot);
 p5 = axes('CreateFcn','imshow(''image/photo_1.jpg'');');
 p = [p1,p2,p3,p4,p5];
 axes_position = [50,290,150,150
                  290,290,150,150
                  530,290,150,150
                  50,60,150,150
                  290,60,150,150];
 
 for i = 1:length(p)
     set(p(i),...
         'Units','pixels',...
         'Position',axes_position(i,:));
 end
 
 % 图片项显示绘图
 function hplot(a,b)
 x = 0:0.1:1;
 y = [-0.447,1.978,3.28,6.16,7.08,7.34,7.66,9.56,9.48,9.3,11.2];
 plot(x,y,'k.','markersize',12);
 hold on;
 axis([0 1.3 -2 16]);
 p3 = polyfit(x,y,3);
 t=0:0.1:1.2;
 s3=polyval(p3,t);
 plot(t,s3,'r');\end{lstlisting}
	\item 测量数据基本处理:subsubpage1.m
	\begin{lstlisting}
 function subsubpage1
 clear
 clc

 %% 创建主界面
 s = get(0,'ScreenSize');% 获取计算机屏幕分辨率
 x = s(3)*0.15;
 y = s(4)*0.26;
 hf = figure('Name','测量数据基本处理',...
 	'NumberTitle','off',...
 	'Position',[x,y,710,450],...
 	'MenuBar','none',...
 	'Color','White',...
 	'Resize','off');

 % 更改界面左上角图标
 icon;

 % 静态文本框
 t = 1:3;
 t_string = {'等精度测量数据误差分析','不等精度测量数据误差分析','返回'};
 t_position = [0,30,300,25;310,30,300,25;620,30,90,25;];
 for i = 1:length(t)
 	t(i) = uicontrol(hf,...
 		'Style','text',...
 		'String',t_string(i),...
 		'FontSize',16,...
 		'FontWeight','bold',...
 		'enable','inactive',...
 		'Units','pixels',...
 		'Position',t_position(i,:));
 end
 set(t(1),'ButtonDownFcn','subsubpage1_1');
 set(t(2),'ButtonDownFcn','subsubpage1_2');
 set(t(3),'ButtonDownFcn','page_exit');

 axes('Units','pixels',...
 	'Position',[0,70,350,400],...
 	'CreateFcn',@plot1);
 axes('Units','pixels',...
 	'Position',[300,70,400,400],...
 	'CreateFcn',@plot2);

 function plot1(a,b)
 [I,~,alpha] = imread('image/photo_1_1.png');
 h = imshow(I);
 set(h,'AlphaData',alpha);

 function plot2(a,b)
 [I,~,alpha] = imread('image/photo_1_2.png');
 h = imshow(I);
 set(h,'AlphaData',alpha);\end{lstlisting}
	\item 等精度测量数据误差分析:subsubpage1\_1.m
	\begin{lstlisting}
 function subsubpage1_1
 clear all;
 clc
 global obj;
 
 %% 创建主界面
 s = get(0,'ScreenSize');% 获取计算机屏幕分辨率
 x = s(3)*0.15;
 y = s(4)*0.26;
 hf = figure('Name','等精度测量数据误差分析',...
     'NumberTitle','off',...
     'Position',[x,y,710,450],...
     'MenuBar','none',...
     'Color','White',...
     'Resize','off');
 
 % 更改界面左上角图标
 icon;
 
 %设置文字项属性
 t = 1:11;
 t_string = {'数据:','置信系数:','平均值:','标准差:','数据:','平均值:',...
 '标准 差:','算术平均值标准差:','结果:','±','残余误差分布图'};
 t_position = [20,405,80,25
                20,350,80,25
 			   20,320,80,25
                20,290,80,25
 			   380,395,80,25
                380,340,80,25
 			   380,310,80,25
                20,260,130,25
                20,10,130,25
                205,10,10,25
                550,250,150,20];
 
 for i = 1:length(t)
     t(i) = uicontrol(hf,...
         'Style','text',...
         'String',t_string(i),...
         'FontName','微软雅黑',...
         'HorizontalAlignment','left',...
         'FontSize',10,...
         'Units','pixels',...
         'Position',t_position(i,:),...
         'BackgroundColor','White');
 end
 
 % 输入数据文本框
 e = 1:10;
 e_position = [100,380,240,50
               100,350,240,25
 			  100,320,240,25
               100,290,240,25
               450,370,240,50
               450,340,240,25
               450,310,240,25
               160,260,180,25
               100,10,100,25
               220,10,100,25];
 
 for i = 1:length(e)
     e(i) = uicontrol(hf,...
         'Style','edit',...
         'FontSize',10,...
         'Units','pixels',...
         'Position',e_position(i,:),...
         'HorizontalAlignment','left',...
         'BackgroundColor','White');
 end
 for i = 3:length(e)
     set(e(i),'Enable','inactive');
 end
 set(e(1),'Min',1,'Max',3);
 set(e(5),'Min',1,'Max',3);
 
 % 面板
 uipanel(...
     'Title','剔除粗大误差后的新数据处理',...
     'FontSize',10,...
 	'FontName','微软雅黑',...
     'Units','pixels',...
     'Position',[350,290,350,160],...
     'BackgroundColor','White');
 
 % 按钮(4)
 b = [uicontrol(hf,'CallBack',@imp),...
      uicontrol(hf,'CallBack',@run1),...
      uicontrol(hf,'CallBack',@outp),...
      uicontrol(hf,'CallBack','page_exit')];
 b_string = {'导入','计算','保存','返回'};
 b_position = [20,380,50,25
               430,10,80,25
               520,10,80,25
               610,10,80,25];
 
 for i = 1:length(b)
     set(b(i),...
         'Style','pushbutton',...
         'String',b_string(i),...
         'FontName','微软雅黑',...
         'FontSize',10,...
         'Units','pixels',...
         'Position',b_position(i,:));
 end
 
 uicontrol(hf,...
     'Style','popup',...
     'Position',[340,10,80,25],...
     'String','数据',...
 	'Value',1,...
 	'CallBack',@data_cho,...
     'FontSize',10);
 
 axes('Units','pixels',...
      'Position',[30,60,650,190],...
      'Box','on');
      
 obj = findobj(gcf);
 
 function data_cho(a,b)
 global data_cell;
 global obj;
 val = get(obj(3),'Value');
 set(obj(18),'String',data_cell{val});
 
 function run1(a,b)
 global obj;
 s1 = str2num(get(obj(18),'String'));
 s2 = str2num(get(obj(17),'String'));
 if isempty(s1)||isempty(s2)
 	warndlg('缺少输入参数!');
 	return;
 end
 [data1,v1,a,a1,s,s1,s1_x,x] = data_process1(s1,s2);
 result = {x(2),x(1),s1_x,s1,a1,data1,s,a};
 axes(obj(2));
 plot(v1,'-o');
 for i = 9:16
     set(obj(i),'String',result{i-8});
 end
 
 function imp(a,b)
 global data_cell;
 global obj;
 [FileName,PathName,FilterIndex] = uigetfile(...
     {'*.txt','Text Data Files(*.txt)';...
      '*.xls','Excel 工作薄(*.xls)'});
 if FileName==0
     return;
 end
 if FilterIndex==1
 	data = load(strcat(PathName,FileName));
 else
 	data = xlsread(strcat(PathName,FileName));
 end
 s = size(data);
 data_cell = cell(s(1),1);
 for i = 1:s(1)
 	a = data(i,:);
 	a(isnan(a)) = [];
 	data_cell{i} = a;
 end
 tip = '数据1';
 for i=2:(s(1))
 	tip = strcat(tip,'|数据',num2str(i));
 end
 set(obj(3),'String',tip);
 set(obj(18),'String',data_cell{1});
 msgbox('导入成功','提示','warn');
 
 function outp(a,b)
 global obj;
 header = {'数据','置信系数','平均值','标准差','剔除粗大误差后平均值',...
 '剔除粗大误差 后标准差','算术平均值标准差','结果'};
 n = [17:-1:15 13:-1:9];
 for i = 1:length(n)
     str{i} = num2str(get(obj(n(i)),'String'));
 end
 a = num2str(get(obj(3),'Value'));
 values = {strcat('数据',a),str{1:6},strcat(str{7},'±',str{8})};
 xlswrite(strcat(datestr(now,30),'.xls'),[header;values]);
 msgbox('保存成功','提示','warn');\end{lstlisting}
	\item 不等精度测量数据误差分析:subsubpage1\_2.m
	\begin{lstlisting}
 function subsubpage1_2
 clear all;
 clc
 global obj;
 
 %% 创建主界面
 s = get(0,'ScreenSize');% 获取计算机屏幕分辨率
 x = s(3)*0.15;
 y = s(4)*0.26;
 hf = figure('Name','不等精度测量数据误差分析',...
     'NumberTitle','off',...
     'Position',[x,y,710,450],...
     'MenuBar','none',...
     'Color','White',...
     'Resize','off');
 
 % 更改界面左上角图标
 icon;
 
 %% 界面控件
 % 设置文字项属性
 t = 1:14;
 t_string = {'数据:','置信系数:','平均值:','标准差:','数据:','平均值:','标准差:',...
 			 '算术平均值标准差:','加权算术平均值:','加权算术平均值标准差:','权:',...
			 '结果:','±','残余误差分布图'};
 t_position = [20,405,80,25
               20,350,80,25
               20,320,80,25
               20,290,80,25
               380,395,80,25
               380,340,80,25
               380,310,80,25
               20,260,130,25
               380,260,130,25
               20,230,180,25
               380,230,180,25
               20,10,130,25
               205,10,10,25
               560,190,100,20];
 
 for i = 1:length(t)
     t(i) = uicontrol(hf,...
         'Style','text',...
         'String',t_string(i),...
         'FontName','微软雅黑',...
         'HorizontalAlignment','left',...
         'FontSize',10,...
         'Units','pixels',...
         'Position',t_position(i,:),...
         'BackgroundColor','White');
 end
 
 % 输入数据文本框
 e = 1:13;
 e_position = [100,380,240,50
               100,350,240,25
 			  100,320,240,25
               100,290,240,25
               450,370,240,50
               450,340,240,25
               450,310,240,25
               160,260,180,25
               450,230,240,25
               500,260,190,25
               180,230,160,25
               100,10,100,25
               220,10,100,25];
 
 for i = 1:length(e)
     e(i) = uicontrol(hf,...
         'Style','edit',...
         'FontSize',10,...
         'Units','pixels',...
         'Position',e_position(i,:),...
         'HorizontalAlignment','left',...
         'BackgroundColor','White');
 end
 for i = 3:length(e)
     set(e(i),'Enable','inactive');
 end
 set(e(1),'Min',1,'Max',3);
 set(e(5),'Min',1,'Max',3);
 
 % 面板
 uipanel(...
     'Title','剔除粗大误差后的新数据处理',...
     'FontSize',10,...
 	'FontName','微软雅黑',...
     'Units','pixels',...
     'Position',[350,290,350,160],...
     'BackgroundColor','White');
 
 % 按钮(4)
 b = [uicontrol(hf,'CallBack',@imp),...
      uicontrol(hf,'CallBack',@run1),...
      uicontrol(hf,'CallBack',@outp),...
      uicontrol(hf,'CallBack','page_exit')];
 b_string = {'导入','计算','保存','返回'};
 b_position = [20,380,50,25
               430,10,80,25
               520,10,80,25
               610,10,80,25];
                  
 for i = 1:length(b)
     set(b(i),...
         'Style','pushbutton',...
         'String',b_string(i),...
         'FontName','微软雅黑',...
         'FontSize',10,...
         'Units','pixels',...
         'Position',b_position(i,:));
 end
 
 uicontrol(hf,...
     'Style','popup',...
     'Position',[340,10,80,25],...
     'String','数据',...
 	'Value',1,...
 	'CallBack',@run1,...
 	'UserData',1,...
     'FontSize',10);
 
 axes('Units','pixels',...
         'Position',[30,60,650,160],...
         'Box','on');
 
 obj = findobj(gcf);
 
 function run1(a,b)
 global data_cell;
 global obj;
 s1 = str2num(get(obj(21),'String'));
 s2 = str2num(get(obj(20),'String'));
 if isempty(s1)||isempty(s2)
 	warndlg('缺少输入参数!');
 	return;
 end
 val = get(obj(3),'Value');
 set(obj(21),'String',data_cell{val});
 [data1,v2,a1,a2,s1,s2,s2_x,p,x_,s_x_,x] = data_process2(data_cell,s2);
 axes(obj(2));
 plot(v2{val},'-o');
 result = {x(2),x(1),s_x_,x_,p(val),s2_x(val),s2(val),a2(val),data1{val},s1(val),a1(val)};
 for i = 9:19
     set(obj(i),'String',result{i-8});
 end
 
 function imp(a,b)
 global data_cell;
 global obj;
 [FileName,PathName,FilterIndex] = uigetfile(...
     {'*.txt','Text Data Files(*.txt)';...
     '*.xls','Excel 工作薄(*.xls)'});
 if FileName==0
     return;
 end
 if FilterIndex==1
 	data = load(strcat(PathName,FileName));
 else
 	data = xlsread(strcat(PathName,FileName));
 end
 s = size(data);
 data_cell = cell(s(1),1);
 for i = 1:s(1)
 	a = data(i,:);
 	a(isnan(a)) = [];
 	data_cell{i} = a;
 end
 tip = '第1组数据';
 for i=2:(s(1))
 	tip = strcat(tip,'|第',num2str(i),'组数据');
 end
 set(obj(3),'String',tip);
 set(obj(21),'String',data_cell{1});
 msgbox('导入成功','提示','warn');
 
 function outp(a,b)
 global obj;
 header = {'数据','置信系数','平均值','标准差','剔除粗大误差后平均值','剔除粗大误差后标准差',...
           '算术平均值标准差','数据的权','加权算术平均值','加权算术平均值标准差','结果'};
 a = num2str(get(obj(3),'Value'));
 n = [20:-1:18 16:-1:9];
 for i = 1:length(n)
     str{i} = get(obj(n(i)),'String');
 end
 values = {strcat('第',a,'组数据'),str{1:9},strcat(str{10},'±',str{11})};
 xlswrite(strcat(datestr(now,30),'.xls'),[header;values]);
 msgbox('保存成功','提示','warn');\end{lstlisting}
	\item 误差的合成:subsubpage2.m
	\begin{lstlisting}
 function subsubpage2
 clear all;
 clc
 global obj;
 
 %% 创建主界面
 s = get(0,'ScreenSize');% 获取计算机屏幕分辨率
 x = s(3)*0.15;
 y = s(4)*0.26;
 hf = figure('Name','误差的合成',...
     'NumberTitle','off',...
     'Position',[x,y,710,450],...
     'MenuBar','none',...
     'Color','White',...
     'Resize','off');
 
 % 更改界面左上角图标
 icon;
 
 % 静态文本框
 t = 1:3;
 t_string = {'a:','Δ:','Result:'};
 t_position = [
     20,420,50,25
     20,290,50,25
     20,160,50,25
 ];
 
 for i = 1:length(t)
     t(i) = uicontrol(hf,...
         'Style','text',...
         'String',t_string(i),...
         'FontName','微软雅黑',...
         'HorizontalAlignment','left',...
         'FontSize',10,...
         'Units','pixels',...
         'Position',t_position(i,:),...
         'BackgroundColor','White');
 end
 
 % 文本框
 e = 1:3;
 e_position = [
     70,320,610,125
     70,190,610,125
     70,60,610,125
 ];
 
 for i = 1:length(e)
     e(i) = uicontrol(hf,...
         'Style','edit',...
         'FontSize',10,...
         'Units','pixels',...
         'Position',e_position(i,:),...
         'BackgroundColor','White',...
         'Min',1,'Max',3,...
         'HorizontalAlignment','left');
 end
 set(e(3),'Enable','inactive');
 
 % 按钮
 b = [uicontrol(hf,'CallBack',@run1),uicontrol(hf,'CallBack','page_exit')];
 b_string = {'计算','返回'};
 b_position = [
     430,10,80,25
     520,10,80,25
 ];
 
 for i = 1:length(b)
     set(b(i),...
         'Style','pushbutton',...
         'String',b_string(i),...
         'FontName','微软雅黑',...
         'FontSize',10,...
         'Units','pixels',...
         'Position',b_position(i,:));
 end
 
 obj = findobj(gcf);
 
 function run1(a,b)
 global obj;
 a = str2num(get(obj(6),'String'));
 delta = str2num(get(obj(5),'String'));
 if isempty(a)||isempty(delta)
 	warndlg('缺少输入参数!');
 	return;
 end
 result = error_combination(a,delta);
 set(obj(4),'String',num2str(result));\end{lstlisting}
	\item 测量不确定度:subsubpage3.m
	\begin{lstlisting}
 function subsubpage3
 clear all
 clc
 global obj;
 
 %% 创建主界面
 s = get(0,'ScreenSize');% 获取计算机屏幕分辨率
 x = s(3)*0.15;
 y = s(4)*0.26;
 hf = figure('Name','测量不确定度',...
     'NumberTitle','off',...
     'Position',[x,y,710,450],...
     'MenuBar','none',...
     'Color','White',...
     'Resize','off');
 
 % 更改界面左上角图标
 icon;
 
 % 面板
 uipanel(...
     'Title','不确定度报告',...
     'FontSize',10,...
 	'FontName','微软雅黑',...
     'Units','pixels',...
     'Position',[20,70,660,230],...
     'BackgroundColor','White');
 
 % 不确定度
 t = 1:18;
 t_string = {'数据:','u1(^-6):','u2(^-6):','u3(^-6):','v1:','v2:','v3:',...
     'V:','u_c:','v:','合成标准不确定度:','展伸不确定度:',...
     'V:','±','P:','v:','置信概率:','包含因子:'};
 t_position = [
     20,420,50,25
     20,370,60,25
     20,340,60,25
     20,310,60,25
     360,370,50,25
     360,340,50,25
     360,310,50,25
     40,220,50,25
     40,190,50,25
     40,160,50,25
     40,250,150,25
     40,120,150,25
     40,90,50,25
     190,90,50,25
     350,90,50,25
     520,90,50,25
     420,420,100,25
     420,395,100,25
 ];
 
 for i = 1:length(t)
     t(i) = uicontrol(hf,...
         'Style','text',...
         'String',t_string(i),...
         'FontName','微软雅黑',...
         'HorizontalAlignment','left',...
         'FontSize',10,...
         'Units','pixels',...
         'Position',t_position(i,:),...
         'BackgroundColor','White');
 end
 
 % 不确定度文本框
 e = 1:16;
 e_position = [
     80,400,320,45
     500,423,160,22
     500,398,160,22
     80,370,240,25
     80,340,240,25
     80,310,240,25
     420,370,240,25
     420,340,240,25
     420,310,240,25
     100,220,550,25
     100,190,550,25
     100,160,550,25
     80,90,110,25
     200,90,110,25
     370,90,110,25
     540,90,110,25
 ];
 for i = 1:length(e)
     e(i) = uicontrol(hf,...
         'Style','edit',...
         'FontName','微软雅黑',...
         'HorizontalAlignment','left',...
         'FontSize',10,...
         'Units','pixels',...
         'Position',e_position(i,:),...
         'BackgroundColor','White');
 end
 for i = 10:length(e)
     set(e(i),'Enable','inactive');
 end
 set(e(1),'Min',1,'Max',3);
 
 % 按钮
 b = [uicontrol(hf,'CallBack',@imp),...
      uicontrol(hf,'CallBack',@run1),...
      uicontrol(hf,'CallBack',@outp),...
      uicontrol(hf,'CallBack','page_exit')];
 b_string = {'导入','计算','保存','返回'};
 b_position = [20,400,50,25
               410,20,80,25
               500,20,80,25
               590,20,80,25];
                  
 for i = 1:length(b)
     set(b(i),...
         'Style','pushbutton',...
         'String',b_string(i),...
         'FontName','微软雅黑',...
         'FontSize',10,...
         'Units','pixels',...
         'Position',b_position(i,:));
 end
 
 uicontrol(hf,...
     'Style','popup',...
     'Position',[310,20,80,25],...
     'String','数据',...
 	'Value',1,...
 	'CallBack',@data_cho,...
 	'UserData',1,...
     'FontSize',10);
 
 obj = findobj(gcf);
 
 function data_cho(a,b)
 global data_cell;
 global obj;
 val = get(obj(2),'Value');
 set(obj(22),'String',data_cell{val});
 
 function imp(a,b)
 global data_cell;
 global obj;
 [FileName,PathName,FilterIndex] = uigetfile(...
     {'*.txt','Text Data Files(*.txt)';...
     '*.xls','Excel 工作薄(*.xls)'});
 if FileName==0
     return;
 end
 if FilterIndex==1
 	data = load(strcat(PathName,FileName));
 else
 	data = xlsread(strcat(PathName,FileName));
 end
 s = size(data);
 data_cell = cell(s(1),1);
 for i = 1:s(1)
 	a = data(i,:);
 	a(isnan(a)) = [];
 	data_cell{i} = a;
 end
 tip = '第1组数据';
 for i=2:(s(1))
 	tip = strcat(tip,'|第',num2str(i),'组数据');
 end
 set(obj(2),'String',tip);
 set(obj(22),'String',data_cell{1});
 msgbox('导入成功','提示','warn');
 
 function run1(a,b)
 global obj;
 for i = 14:22
     s = str2num(get(obj(i),'String'));
     if isempty(s)
 	    warndlg('缺少输入参数!');
 	    return;
     end
 end
 V = str2num(get(obj(22),'String'));
 n1 = 21:-1:14;
 for i=1:length(n1);
     x(i) = str2num(get(obj(n1(i)),'String'));
 end
 [V_,u_c,v_,U,P_] = uncertainty(V,x(3:5),x(6:8),x(1),x(2));
 result = [v_,P_,U/1000000,V_,v_,u_c/1000000,V_];
 for i=7:13
     set(obj(i),'String',result(i-6));
 end
 
 function outp(a,b)
 global obj;
 header = {'数据:','u1(^-6)','u2(^-6)','u3(^-6)','v1','v2','v3',...
         'V','u_c','v','V','P','v','置信概率','包含因子'};
 n = [19:-1:7 21:-1:20];
 for i = 1:length(n)
     str{i} = num2str(get(obj(n(i)),'String'));
 end
 a = num2str(get(obj(2),'Value'));
 values = {strcat('数据',a),str{1:9},strcat(str{10},'±',str{11}),str{12:15}};
 xlswrite(strcat(datestr(now,30),'.xls'),[header;values]);
 msgbox('保存成功','提示','warn');\end{lstlisting}
	\item 最小二乘法处理:subsubpage4.m
	\begin{lstlisting}
 function subsubpage4
 clear
 clc
 %% 创建主界面
 s = get(0,'ScreenSize');% 获取计算机屏幕分辨率
 
 x = s(3)*0.15;
 y = s(4)*0.26;
 hf = figure('Name','最小二乘法处理',...
     'NumberTitle','off',...
    'Position',[x,y,710,450],...
    'MenuBar','none',...
    'Color','White',...
    'Resize','off');
 % 更改界面左上角图标
 icon;

 % 静态文本框
 t = 1:3;
 
 t_string = {'等精度测量线性参数最小二乘法处理','不等精度测量线性参数最小二乘法处理','返回'};
 t_position = [0,100,400,25;210,60,400,25;360,20,350,25;];
 for i = 1:length(t)
     t(i) = uicontrol(hf,...
        'Style','text',...
        'String',t_string(i),...
        'FontSize',16,...
        'FontWeight','bold',...
        'enable','inactive',...
        'Units','pixels',...
        'Position',t_position(i,:));
 end
 set(t(1),'ButtonDownFcn','subsubpage4_1');
 set(t(2),'ButtonDownFcn','subsubpage4_2');
 set(t(3),'ButtonDownFcn','page_exit');
 t1 = 1:2;
 t1_string = {'图一:最小二乘法处理数据','图二:最小二乘法计算玉米粒数'};

 t1_position = [0,385,350,25;361,385,350,25;];
 for i = 1:length(t1)
     t1(i) = uicontrol(hf,...
         'Style','text',...
        'String',t1_string(i),...
        'FontSize',10,...
        'BackgroundColor','white',...
        'Units','pixels',...
        'Position',t1_position(i,:));
 end
 axes('Units','pixels',...
    'Position',[0,70,350,400],...

     'CreateFcn','imshow(''image/photo_4_1.jpg'');');
 axes('Units','pixels',...
     'Position',[361,70,350,400],...
     'CreateFcn','imshow(''image/photo_4_2.jpg'');');\end{lstlisting}
 	\item 等精度测量线性参数最小二乘法处理:subsubpage4\_1.m
 	\begin{lstlisting}
 function subsubpage4_1
 clear all;
 clc
 global obj;
 
 %% 创建主界面
 s = get(0,'ScreenSize');% 获取计算机屏幕分辨率
 x = s(3)*0.15;
 y = s(4)*0.26;
 hf = figure('Name','等精度测量线性参数最小二乘法处理',...
     'NumberTitle','off',...
     'Position',[x,y,710,450],...
     'MenuBar','none',...
     'Color','White',...
     'Resize','off');
 
 % 更改界面左上角图标
 icon;
 
 % 静态文本框
 t = 1:8;
 t_string = {'A:','L:','逆矩阵结果:','最小二乘估计结果:','残余误差结果:','残余平方和:',...
 '单次测量标准差:','待估计量相应的标准差:'};
 t_position = [
     20,420,50,25
     360,420,50,25
     20,290,100,25
     360,290,120,25
     20,160,100,25
     360,160,120,25
     360,130,120,25
     360,100,150,25
 ];
 
 for i = 1:length(t)
     t(i) = uicontrol(hf,...
         'Style','text',...
         'String',t_string(i),...
         'FontName','微软雅黑',...
         'HorizontalAlignment','left',...
         'FontSize',10,...
         'Units','pixels',...
         'Position',t_position(i,:),...
         'BackgroundColor','White');
 end
 
 % 文本框
 e = 1:8;
 e_position = [
     70,320,280,125
     410,320,280,125
     120,190,230,125
     480,190,210,125
     120,60,230,125
     480,160,210,25
     480,130,210,25
     510,60,180,65
 ];
 
 for i = 1:length(e)
     e(i) = uicontrol(hf,...
         'Style','edit',...
         'FontSize',10,...
         'Units','pixels',...
         'Position',e_position(i,:),...
         'BackgroundColor','White',...
         'Min',1,'Max',3,...
         'HorizontalAlignment','left');
 end
 for i = 3:length(e)
     set(e(i),'Enable','inactive');
 end
 set(e(6),'Max',1);
 set(e(7),'Max',1);
 
 % 按钮
 b = [uicontrol(hf,'CallBack',@run1),uicontrol(hf,'CallBack','page_exit')];
 b_string = {'计算','返回'};
 b_position = [
     430,10,80,25
     520,10,80,25
 ];
 
 for i = 1:length(b)
     set(b(i),...
         'Style','pushbutton',...
         'String',b_string(i),...
         'FontName','微软雅黑',...
         'FontSize',10,...
         'Units','pixels',...
         'Position',b_position(i,:));
 end
 
 obj = findobj(gcf);
 
 function run1(a,b)
 global obj;
 A = str2num(get(obj(11),'String'));
 L = str2num(get(obj(10),'String'));
 if isempty(A)||isempty(L)
 	warndlg('缺少输入参数!');
 	return;
 end
 [D,EX,V,V_,s,d_ux] = data_process3(A,L);
 result = {d_ux,s,V_,V,EX,D};
 for i=5:10
     set(obj(i),'String',num2str(result{i-4}));
 end\end{lstlisting}
	\item 不等精度测量线性参数最小二乘法处理:subsubpage4\_2.m
	\begin{lstlisting}
 function subsubpage4_2
 clear all;
 clc
 global obj;
 
 %% 创建主界面
 s = get(0,'ScreenSize');% 获取计算机屏幕分辨率
 x = s(3)*0.15;
 y = s(4)*0.26;
 hf = figure('Name','不等精度测量线性参数最小二乘法处理',...
     'NumberTitle','off',...
     'Position',[x,y,710,450],...
     'MenuBar','none',...
     'Color','White',...
     'Resize','off');
 
 % 更改界面左上角图标
 icon;
 
 % 静态文本框
 t = 1:6;
 t_string = {'A:','L:','P:','中间结果:','逆矩阵结果:','最小二乘估计结果:'};
 t_position = [
     20,420,50,25
     20,290,50,25
     20,160,50,25
     360,420,120,25
     360,290,100,25
     360,160,120,25
 ];
 
 for i = 1:length(t)
     t(i) = uicontrol(hf,...
         'Style','text',...
         'String',t_string(i),...
         'FontName','微软雅黑',...
         'HorizontalAlignment','left',...
         'FontSize',10,...
         'Units','pixels',...
         'Position',t_position(i,:),...
         'BackgroundColor','White');
 end
 
 % 文本框
 e = 1:6;
 e_position = [
     70,320,260,125
     70,190,260,125
     70,60,260,125
     480,320,200,125
     480,190,200,125
     480,60,200,125
 ];
 
 for i = 1:length(e)
     e(i) = uicontrol(hf,...
         'Style','edit',...
         'FontSize',10,...
         'Units','pixels',...
         'Position',e_position(i,:),...
         'BackgroundColor','White',...
         'Min',1,'Max',3,...
         'HorizontalAlignment','left');
 end
 for i = 4:length(e)
     set(e(i),'Enable','inactive');
 end
 
 % 按钮
 b = [uicontrol(hf,'CallBack',@run1),uicontrol(hf,'CallBack','page_exit')];
 b_string = {'计算','返回'};
 b_position = [
     430,10,80,25
     520,10,80,25
 ];
 
 for i = 1:length(b)
     set(b(i),...
         'Style','pushbutton',...
         'String',b_string(i),...
         'FontName','微软雅黑',...
         'FontSize',10,...
         'Units','pixels',...
         'Position',b_position(i,:));
 end
 
 obj = findobj(gcf);
 
 function run1(a,b)
 global obj;
 A = str2num(get(obj(9),'String'));
 L = str2num(get(obj(8),'String'));
 P = str2num(get(obj(7),'String'));
 if isempty(A)||isempty(L)||isempty(P)
 	warndlg('缺少输入参数!');
 	return;
 end
 [C,D,EX] = data_process4(A,L,P);
 set(obj(7),'String',num2str(C));
 set(obj(6),'String',num2str(D));
 set(obj(5),'String',num2str(EX));\end{lstlisting}
	\item 回归分析:subsubpage5.m
	\begin{lstlisting}
 function subsubpage5
 clear all;
 clc
 global obj;
 
 %% 创建主界面
 s = get(0,'ScreenSize');% 获取计算机屏幕分辨率
 x = s(3)*0.15;
 y = s(4)*0.26;
 hf = figure('Name','回归分析',...
     'NumberTitle','off',...
     'Position',[x,y,710,450],...
     'MenuBar','none',...
     'Color','White',...
     'Resize','off');
 
 % 更改界面左上角图标
 icon;
 
 % 静态文本框
 t = 1:5;
 t_string = {'x:','y:','n次多项式拟合:','残差分析结果:','拟合结果:'};
 t_position = [
     20,420,50,25
     20,390,50,25
     20,360,100,25
     20,330,100,25
     360,330,100,25
 ];
 
 for i = 1:length(t)
     t(i) = uicontrol(hf,...
         'Style','text',...
         'String',t_string(i),...
         'FontName','微软雅黑',...
         'HorizontalAlignment','left',...
         'FontSize',10,...
         'Units','pixels',...
         'Position',t_position(i,:),...
         'BackgroundColor','White');
 end
 
 % 文本框
 e = 1:2;
 e_position = [
     70,420,600,25
     70,390,600,25
 ];
 
 for i = 1:length(e)
     e(i) = uicontrol(hf,...
         'Style','edit',...
         'FontSize',10,...
         'Units','pixels',...
         'Position',e_position(i,:),...
         'BackgroundColor','White',...
         'Min',1,'Max',3,...
         'HorizontalAlignment','left');
 end
 
 uicontrol(hf,...
     'Style','popup',...
     'Position',[130,360,100,25],...
     'String','1|2|3|4|5|6|7|8',...
 	'Value',2,...
     'FontSize',10,...
     'BackgroundColor','White');
 
 % 按钮
 b = [uicontrol(hf,'CallBack',@imp),...
      uicontrol(hf,'CallBack',@run1),...
      uicontrol(hf,'CallBack',@outp),...
      uicontrol(hf,'CallBack','page_exit')];
 b_string = {'导入','计算','保存','返回'};
 b_position = [
     250,360,80,25
     430,10,80,25
     520,10,80,25
     610,10,80,25
 ];
 
 for i = 1:length(b)
     set(b(i),...
         'Style','pushbutton',...
         'String',b_string(i),...
         'FontName','微软雅黑',...
         'FontSize',10,...
         'Units','pixels',...
         'Position',b_position(i,:));
 end
 
 axes('Units','pixels',...
     'Position',[30,60,300,260],...
     'Box','on');
 
 axes('Units','pixels',...
     'Position',[370,60,300,260],...
     'Box','on');
 
 obj = findobj(gcf);
 
 function imp(a,b)
 global obj;
 [FileName,PathName,FilterIndex] = uigetfile(...
     {'*.txt','Text Data Files(*.txt)';...
      '*.xls','Excel 工作薄(*.xls)'});
 if FileName==0
     return;
 end
 if FilterIndex==1
 	data = load(strcat(PathName,FileName));
 else
 	data = xlsread(strcat(PathName,FileName));
 end
 set(obj(10),'String',data(1,:));
 set(obj(9),'String',data(2,:));
 msgbox('导入成功','提示','warn');
 
 function run1(a,b)
 global obj;
 x = str2num(get(obj(10),'String'));
 y = str2num(get(obj(9),'String'));
 l = size(x);
 if l(1)>1
     x = x';
     y = y';
 end
 n = get(obj(8),'Value');
 if isempty(x)||isempty(y)
 	warndlg('缺少输入参数!');
 	return;
 end
 I = ones(length(x),1);
 for i=1:n
     b = x.^i;
     I = [I b'];
 end
 [b,bint,r,rint,stats]=regress(y',I);
 axes(obj(3));
 rcoplot(r,rint);
 title('');
 xlabel('');
 ylabel('');
 Y = polyval(b(end:-1:1),x);
 axes(obj(2));
 plot(x,y,'k+',x,Y,'r');
 
 function outp(a,b)
 global obj;
 f1 = getframe(obj(3));
 f2 = getframe(obj(2));
 imwrite(f1.cdata,'残差分析结果.jpg','jpg')
 imwrite(f2.cdata,'拟合结果.jpg','jpg')
 msgbox('保存图像成功','提示','warn');\end{lstlisting}
	\item 返回上一级:subsubpage6.m
	\begin{lstlisting}
 %%返回上一级菜单
 page_exit\end{lstlisting}
\end{enumerate}
\newpage
\begin{center}
	\textbf{附录4 误差理论与数据处理算法代码}
\end{center}
\begin{enumerate}
	\item 逖克逊准则粗大误差处理:BlodBig.m
	\begin{lstlisting}
 function data1 = BlodBig(data)
 % 临界值 a = 0.05
 r0 = [0 0 0.941 0.765 0.642 0.560 0.507 0.554 0.512 0.447 0.576 0.546 0.521 0.548 0.525 0.507 0.490 0.475 0.462 0.450 0.440 0.430 0.421  0.413 0.406 0.399 0.393 0.387 0.381 0.378];
 
 n = length(data);
 data_ = sort(data);
 if n<3
     msgbox('数据太少','提示','warn');
     data1 = data;
     return
 elseif n>=3 && n<=7
     r = (data_(n)-data_(n-1))/(data_(n)-data_(1));
     r_ = (data_(1)-data_(2))/(data_(1)-data_(n));
 elseif n>=8 && n<=10
     r = (data_(n)-data_(n-1))/(data_(n)-data_(2));
     r_ = (data_(1)-data_(2))/(data_(1)-data_(n-1));
 elseif n>=11 && n<=13
     r = (data_(n)-data_(n-2))/(data_(n)-data_(2));
     r_ = (data_(1)-data_(3))/(data_(1)-data_(n-1));
 else
     r = (data_(n)-data_(n-2))/(data_(n)-data_(3));
     r_ = (data_(1)-data_(3))/(data_(1)-data_(n-2));
 end
 if r>=r0(n)
     data(data==data_(n)) = [];
 elseif r_>=r0(n)
     data(data==data_(1)) = [];
 end
 data1 = data;\end{lstlisting}
	\item 等精度测量数据误差分析:data\_process1.m
	\begin{lstlisting}
 function [data1,v1,a,a1,s,s1,s1_x,x] = data_process1(data,t_a)
 a = mean(data);
 s = std(data);
 data1 = BlodBig(data);
 
 a1 = mean(data1);
 s1 = std(data1);
 n1 = length(data1);
 v1 = 1:n1;
 for i=1:n1
     v1(i) = data1(i)-a1;
 end
 s1_x = s1/sqrt(n1);
 sigama = t_a*s1_x;
 x = roundn([a1 sigama],-3);\end{lstlisting}
	\item 不等精度测量数据误差分析:data\_process2.m
	\begin{lstlisting}
 function [data1,v2,a1,a2,s1,s2,s2_x,p,x_,s_x_,x] = data_process2(data,t_a)
 s = size(data);
 for i = 1:s(1)
     a1(i) = mean(data{i});
     s1(i) = std(data{i});
     data1{i} = BlodBig(data{i});
 end
 
 for i = 1:s(1)
     a2(i) = mean(data1{i});
     s2(i) = std(data1{i});
     s2_x(i) = s2(i)/sqrt(length(data1{i}));
 end
 % 残差
 n2 = size(data1);
 for i=1:n2(2)
     a = [];
     b = data1{i};
     for j=1:length(b)
         a(j) = b(j)-a2(i);
     end
     v2{i} = a;
 end
 % 权
 for i = 1:s(1)
     p(i) = 1/(s2_x(i)*s2_x(i));
 end
 % 加权算术平均值
 [x_,s_x] = jiaquan(p,a2);
 s_x_ = s2_x(1)*s_x;
 
 sigama = t_a*s_x_;
 x = roundn([x_ sigama],-3);
 
 function [x_,s_x_] = jiaquan(p,x_)
 n = length(p);
 s1 = 0;
 s2 = 0;
 for i = 1:n
     s1 = s1+p(i)*x_(i);
     s2 = s2+p(i);
 end
 x_ = s1/s2;
 s_x_ = sqrt(p(1)/s2);\end{lstlisting}
	\item 等精度测量线性参数最小二乘法处理:data\_process3.m
	\begin{lstlisting}
 function [D,EX,V,V_,s,d_ux] = data_process3(A,L)
 B = A';
 C = B*A;
 D = inv(C);
 EX = D*B*L;
 V = L-A*EX;
 V_ = V'*V;
 
 s = size(A);
 n = s(1);
 t = s(2);
 s = sqrt(V'*V./(n-t));
 
 d = 1:t;
 ux = 1:t;
 for i = 1:t
     d(i) = D(i,i);
     ux(i) = s*sqrt(d(i));
 end
 d_ux = [d' ux'];\end{lstlisting}
	\item 不等精度测量线性参数最小二乘法处理:data\_process4.m
	\begin{lstlisting}
 function [C,D,EX] = data_process4(A,L,P)
 C = A'*P*A;
 D = inv(C);
 EX = D*A'*P*L;\end{lstlisting}
	\item 误差的合成:error\_combination.m
	\begin{lstlisting}
 function y = error_combination(a,delta)
 na = length(a);
 nd = length(delta);
 if na~=nd
     msgbox('数据长度不一样!','提示','warn');
     return;
 end
 y = 0;
 for i=1:na
     y = y + a(i)*delta(i);
 end\end{lstlisting}
	\item 测量不确定度:uncertainty.m
	\begin{lstlisting}
 function [V_,u_c,v_,U,P_] = uncertainty(V,u,v,P,k)
 V_ = mean(V);
 u_c = roundn(sqrt(u(1)^2+u(2)^2+u(3)^3),0);
 v_ = floor((u_c^4)/((u(1)^4)/v(1)+(u(2)^4)/v(2)+(u(3)^4)/v(3)));
 U = ceil(k*u_c);
 P_ = P;\end{lstlisting}
\end{enumerate}