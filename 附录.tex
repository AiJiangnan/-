\part*{附录}
\addcontentsline{toc}{part}{附录}
\begin{center}
	\textbf{附录A 狄克逊检验统计量和临界值}
\end{center}

\centering
\begin{tabular}{|p{4cm}<{\centering}|p{2cm}<{\centering}|p{2cm}<{\centering}|p{2cm}<{\centering}|p{2cm}<{\centering}|}
	\hline
	统计量&$ n $& 0.10&\tabincell{c}{$\alpha$\\0.05 }& 0.01 \\	\hline
	\multirow{5}*{\tabincell{l}{$ r_{10}=\frac{x_{(n)}-x_{(n-1)}}{x_{(n)}-x_{(1)}}$\\$r'_{10}=\frac{x_{(1)}-x_{(2)}}{x_{(1)}-x_{(n)}} $}}														&3	&0.886	&0.941	&0.988	\\	\cline{2-5}
	&4	&0.679	&0.765	&0.889	\\	\cline{2-5}
	&5	&0.557	&0.642	&0.780	\\	\cline{2-5}
	&6	&0.482	&0.560	&0.698	\\	\cline{2-5}
	&7	&0.434	&0.507	&0.637	\\	\hline
	\multirow{3}*{\tabincell{l}{$ r_{11}=\frac{x_{(n)}-x_{(n-1)}}{x_{(n)}-x_{(2)}} $\\$ r'_{11}=\frac{x_{(1)}-x_{(2)}}{x_{(1)}-x_{(n-1)}} $}}														&8	&0.479	&0.554	&0.683	\\	\cline{2-5}
	&9	&0.441	&0.512	&0.635	\\	\cline{2-5}
	&10	&0.409	&0.447	&0.597	\\	\hline
	\multirow{3}*{\tabincell{l}{$ r_{21}=\frac{x_{(n)}-x_{(n-2)}}{x_{(n)}-x_{(2)}} $\\$ r'_{21}=\frac{x_{(1)}-x_{(3)}}{x_{(1)}-x_{(n-1)}} $}}														&11	&0.517	&0.576	&0.670	\\	\cline{2-5}
	&12	&0.490	&0.546	&0.642	\\	\cline{2-5}
	&13	&0.467	&0.521	&0.615	\\	\hline
	\multirow{17}*{\tabincell{l}{$ r_{22}=\frac{x_{(n)}-x_{(n-2)}}{x_{(n)}-x_{(3)}} $\\$ r'_{22}=\frac{x_{(1)}-x_{(3)}}{x_{(1)}-x_{(n-2)}} $}}														&14	&0.492	&0.548	&0.641	\\	\cline{2-5}
	&15	&0.472	&0.525	&0.616	\\	\cline{2-5}
	&16	&0.454	&0.507	&0.595	\\	\cline{2-5}
	&17	&0.438	&0.490	&0.577	\\	\cline{2-5}
	&18	&0.424	&0.475	&0.561	\\	\cline{2-5}
	&19	&0.412	&0.462	&0.547	\\	\cline{2-5}
	&20	&0.401	&0.450	&0.535	\\	\cline{2-5}
	&21	&0.391	&0.440	&0.524	\\	\cline{2-5}
	&22	&0.382	&0.430	&0.514	\\	\cline{2-5}
	&23	&0.374	&0.421	&0.505	\\	\cline{2-5}
	&24	&0.367	&0.413	&0.497	\\	\cline{2-5}
	&25	&0.360	&0.406	&0.489	\\	\cline{2-5}
	&26	&0.354	&0.399	&0.486	\\	\cline{2-5}
	&27	&0.348	&0.393	&0.475	\\	\cline{2-5}
	&28	&0.342	&0.387	&0.469	\\	\cline{2-5}
	&29	&0.337	&0.381	&0.463	\\	\cline{2-5}
	&30	&0.332	&0.378	&0.457	\\	\hline
\end{tabular}
\newpage
\begin{center}
	\textbf{附录B GUI设计函数常用属性\footnote{参考文献[2]第9章Matlab GUI的组成与结构第196页}}
\end{center}
\begin{tabular}{|p{3cm}|p{5cm}|p{5cm}|}
	\hline
	\multicolumn{1}{|c|}{\textbf{对象属性}}	&\multicolumn{1}{c}{\textbf{意义}}	&\multicolumn{1}{|c|}{\textbf{取值及含义}}\\	\hline
	BackgroundColor	&uicontrol背景色。3元素的RGB向量或Matlab预先定义的颜色名称	&默认的背景色是浅灰色		\\	\hline
	Callback		&Matlab回调串,当uicontrol激活时,回调串传给函数eval	   &初始值为空矩阵			\\	\hline
	Max				&属性Value的最大许可值									 &默认值为1				\\	\hline
	Min				&属性Value的最小许可值									 &默认值为0				\\	\hline
	Position		&位置向量[left bottom width height]					   	&					    \\	\hline
	String			&文本字符串,uicontrol显示的文本						    &					 \\	\hline
	Style			&定义uicontrol的类型					&text,edit,pushbutton...				\\	\hline
	Units			&设置属性值的单位										&pixels:屏幕像素		\\	\hline
	Value			&uicontrol的当前值										&文本对象和按钮不设置该属性	\\	\hline
	UserData		&用户指定的数据。可以是矩阵、字符串等						&						\\	\hline
	Tag				&文本串												&						\\	\hline
	CreateFcn		&对象创建时执行的回调函数								&						\\	\hline
	DeleteFcn		&对象删除时执行的回调函数								&						\\	\hline
	CloseRequestFcn		&窗口关闭时执行的回调函数							&						\\	\hline
	ButtonDownFcn		&对象鼠标单击时执行的回调函数						  &						\\	\hline
\end{tabular}
\newpage
\begin{center}
	\textbf{附录C 界面设计代码}
\end{center}
\begin{enumerate}
	\item 设置界面图标:icon.m
	\begin{lstlisting}
 % 更改界面左上角图标
 newIcon = javax.swing.ImageIcon('./image/logo.png');
 figFrame = get(gcf,'JavaFrame');
 figFrame.setFigureIcon(newIcon);\end{lstlisting}
	\item 界面返回:page\_exit.m
	\begin{lstlisting}
 %% 返回上一级菜单
 close(gcf);
 clear all
 clc\end{lstlisting}
 	\item 主界面:index.m
	\begin{lstlisting}
 function index
 clear
 clc
 
 % 添加子文件夹路径
 addpath(genpath(pwd));
 
 %% 创建主界面
 s = get(0,'ScreenSize');% 获取计算机屏幕分辨率
 x = s(3)*0.15;
 y = s(4)*0.26;
 hf = figure('Name','基于GUI的误差理论与数据处理系统',...
     'NumberTitle','off',...
     'Units','pixels',...
     'Position',[x,y,600,450],...
     'MenuBar','none',...
     'Color','White',...
     'CloseRequestFcn',@hexit,...
     'Resize','off');
 
 % 更改界面左上角图标
 icon;
 
 % 学校名称
 axes('Units','pixels',...
     'Position',[0,350,430,100],...
     'CreateFcn',@school_logo);
 
 % 校训
 uicontrol(hf,...
     'Units','pixels',...
     'Position',[300,320,230,25],...
     'Style','text',...
     'String','崇德博智,扶危定倾',...
     'FontName','楷体',...
     'FontSize',18,...
     'FontWeight','bold',...
     'ForegroundColor',[7,51,123]/255,...
     'BackgroundColor','White');
 
 % 学校标志性建筑
 axes('Units','pixels',...
     'Position',[0,80,600,233],...
     'CreateFcn','imshow(''image/school.jpg'')');
 
 %% 一级菜单按钮
 uicontrol(hf,...
     'Units','pixels',...
     'Position',[160,35,280,25],...
     'Style','text',...
     'String','《误差理论与数据处理》',...
     'FontName','楷体',...
     'FontSize',18,...
     'FontWeight','bold',...
     'ForegroundColor',[7,51,123]/255,...
     'BackgroundColor','White',...
     'enable','inactive',...
     'ButtonDownFcn','subpage');
 uicontrol(hf,...
     'Units','pixels',...
     'Position',[160,5,280,25],...
     'Style','text',...
     'String','Error Theory and Data Processing',...
     'FontSize',12,...
     'FontWeight','bold',...
     'ForegroundColor',[7,51,123]/255,...
     'BackgroundColor','White',...
     'enable','inactive',...
     'ButtonDownFcn','subpage');
 
 % 显示学校名称PNG图片函数
 function school_logo(a,b)
 [I,c,alpha] = imread('image/school_logo.png');
 h = imshow(I);
 set(h,'AlphaData',alpha);
 
 %% 主界面退出对话框
 function hexit(a,b)
 he = questdlg('你确定退出吗?','退出程序','是','否','否');
 if strcmp(he,'是')
     close;
     clear;
     clc;
 end;\end{lstlisting}
 \newpage
	\item 二级子界面:subpage.m
	\begin{lstlisting}
 function subpage
 clear
 clc
 
 %% 创建主界面
 s = get(0,'ScreenSize');% 获取计算机屏幕分辨率
 x = s(3)*0.15;
 y = s(4)*0.26;
 hf = figure('Name','基于GUI的误差理论与数据处理系统',...
     'NumberTitle','off',...
     'Units','pixels',...
     'Position',[x,y,710,450],...
     'MenuBar','none',...
     'Color','White',...
     'Resize','off');
 
 % 更改界面左上角图标
 icon;
 
 %创建文字项
 t = 1:6;
 menu_string = {'测量数据基本处理','误差的合成','测量不确定度','最小二乘法处理',' 回归分析','返回'};
 menu_position = [0,240,230,25
                  240,240,230,25
                  480,240,230,25
                  0,10,230,25
                  240,10,230,25
                  480,10,230,25];
 %设置文字项属性
 for i = 1:length(t)
     t(i) = uicontrol(hf,...
         'Style','text',...
         'String',menu_string(i),...
         'FontName','微软雅黑',...
         'FontSize',14,...
         'FontWeight','bold',...
         'enable','inactive',...
         'Units','pixels',...
         'Position',menu_position(i,:),...
         'ButtonDownFcn',strcat('subsubpage',num2str(i)));
 end
 
 %创建图片项
 p1 = axes('CreateFcn','bar([2,1,3,5,3]);');
 p2 = axes('CreateFcn','imshow(''image/photo_2.jpg'');');
 p3 = axes('CreateFcn','imshow(''image/photo_3.jpg'');');
 p4 = axes('CreateFcn',@hplot);
 p5 = axes('CreateFcn','imshow(''image/photo_1.jpg'');');
 p = [p1,p2,p3,p4,p5];
 axes_position = [50,290,150,150
                  290,290,150,150
                  530,290,150,150
                  50,60,150,150
                  290,60,150,150];
 
 for i = 1:length(p)
     set(p(i),...
         'Units','pixels',...
         'Position',axes_position(i,:));
 end
 
 % 图片项显示绘图
 function hplot(a,b)
 x = 0:0.1:1;
 y = [-0.447,1.978,3.28,6.16,7.08,7.34,7.66,9.56,9.48,9.3,11.2];
 plot(x,y,'k.','markersize',12);
 hold on;
 axis([0 1.3 -2 16]);
 p3 = polyfit(x,y,3);
 t=0:0.1:1.2;
 s3=polyval(p3,t);
 plot(t,s3,'r');\end{lstlisting}
\end{enumerate}
\newpage
\begin{center}
	\textbf{附录D 误差理论与数据处理算法代码}
\end{center}
\begin{enumerate}
	\item 逖克逊准则粗大误差处理:BlodBig.m
	\begin{lstlisting}
 function data1 = BlodBig(data)
 % 临界值 a = 0.05
 r0 = [0 0 0.941 0.765 0.642 0.560 0.507 0.554 0.512 0.447 0.576 0.546 0.521 0.548 0.525 0.507 0.490 0.475 0.462 0.450 0.440 0.430 0.421  0.413 0.406 0.399 0.393 0.387 0.381 0.378];
 
 n = length(data);
 data_ = sort(data);
 if n<3
     msgbox('数据太少','提示','warn');
     data1 = data;
     return
 elseif n>=3 && n<=7
     r = (data_(n)-data_(n-1))/(data_(n)-data_(1));
     r_ = (data_(1)-data_(2))/(data_(1)-data_(n));
 elseif n>=8 && n<=10
     r = (data_(n)-data_(n-1))/(data_(n)-data_(2));
     r_ = (data_(1)-data_(2))/(data_(1)-data_(n-1));
 elseif n>=11 && n<=13
     r = (data_(n)-data_(n-2))/(data_(n)-data_(2));
     r_ = (data_(1)-data_(3))/(data_(1)-data_(n-1));
 else
     r = (data_(n)-data_(n-2))/(data_(n)-data_(3));
     r_ = (data_(1)-data_(3))/(data_(1)-data_(n-2));
 end
 if r>=r0(n)
     data(data==data_(n)) = [];
 elseif r_>=r0(n)
     data(data==data_(1)) = [];
 end
 data1 = data;\end{lstlisting}
	\item 等精度测量数据误差分析:data\_process1.m
	\begin{lstlisting}
 function [data1,v1,a,a1,s,s1,s1_x,x] = data_process1(data,t_a)
 a = mean(data);
 s = std(data);
 data1 = BlodBig(data);
 
 a1 = mean(data1);
 s1 = std(data1);
 n1 = length(data1);
 v1 = 1:n1;
 for i=1:n1
     v1(i) = data1(i)-a1;
 end
 s1_x = s1/sqrt(n1);
 sigama = t_a*s1_x;
 x = roundn([a1 sigama],-3);\end{lstlisting}
	\item 不等精度测量数据误差分析:data\_process2.m
	\begin{lstlisting}
 function [data1,v2,a1,a2,s1,s2,s2_x,p,x_,s_x_,x] = data_process2(data,t_a)
 s = size(data);
 for i = 1:s(1)
     a1(i) = mean(data{i});
     s1(i) = std(data{i});
     data1{i} = BlodBig(data{i});
 end
 
 for i = 1:s(1)
     a2(i) = mean(data1{i});
     s2(i) = std(data1{i});
     s2_x(i) = s2(i)/sqrt(length(data1{i}));
 end
 % 残差
 n2 = size(data1);
 for i=1:n2(2)
     a = [];
     b = data1{i};
     for j=1:length(b)
         a(j) = b(j)-a2(i);
     end
     v2{i} = a;
 end
 % 权
 for i = 1:s(1)
     p(i) = 1/(s2_x(i)*s2_x(i));
 end
 % 加权算术平均值
 [x_,s_x] = jiaquan(p,a2);
 s_x_ = s2_x(1)*s_x;
 
 sigama = t_a*s_x_;
 x = roundn([x_ sigama],-3);
 
 function [x_,s_x_] = jiaquan(p,x_)
 n = length(p);
 s1 = 0;
 s2 = 0;
 for i = 1:n
     s1 = s1+p(i)*x_(i);
     s2 = s2+p(i);
 end
 x_ = s1/s2;
 s_x_ = sqrt(p(1)/s2);\end{lstlisting}
 \newpage
	\item 等精度测量线性参数最小二乘法处理:data\_process3.m
	\begin{lstlisting}
 function [D,EX,V,V_,s,d_ux] = data_process3(A,L)
 B = A';
 C = B*A;
 D = inv(C);
 EX = D*B*L;
 V = L-A*EX;
 V_ = V'*V;
 
 s = size(A);
 n = s(1);
 t = s(2);
 s = sqrt(V'*V./(n-t));
 
 d = 1:t;
 ux = 1:t;
 for i = 1:t
     d(i) = D(i,i);
     ux(i) = s*sqrt(d(i));
 end
 d_ux = [d' ux'];\end{lstlisting}
	\item 不等精度测量线性参数最小二乘法处理:data\_process4.m
	\begin{lstlisting}
 function [C,D,EX] = data_process4(A,L,P)
 C = A'*P*A;
 D = inv(C);
 EX = D*A'*P*L;\end{lstlisting}
\end{enumerate}